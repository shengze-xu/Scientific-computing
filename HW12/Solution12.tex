\documentclass{article}

\usepackage{fancyhdr}
\usepackage[toc,page]{appendix}
\usepackage{extramarks}
\usepackage{amsmath}
\usepackage{amsthm}
\usepackage{amsfonts}
\usepackage{tikz}
\usepackage[plain]{algorithm}
\usepackage{algpseudocode}
\usepackage{amssymb}
\usepackage{bbm}
\usepackage{extarrows}
\usepackage{mathrsfs}
\usepackage{CJK}
\usepackage{dsfont}
\usepackage[hidelinks]{hyperref}
\usepackage{apacite}
\usepackage{multirow, booktabs}  
\usepackage{threeparttable}
\usepackage{dcolumn}
\usepackage{longtable}
\usepackage{threeparttablex}
\usepackage{tabu}
\usepackage{pdfpages}
\usepackage{float}
\usepackage{changepage}
\usepackage{mathtools}
\usepackage{listings}

\usetikzlibrary{automata,positioning}

\topmargin=-0.45in
\evensidemargin=0in
\oddsidemargin=0in
\textwidth=6.5in
\textheight=9.0in
\headsep=0.25in
\setlength{\parindent}{0em}
\linespread{1.1}

\pagestyle{myheadings}
\markboth{HW Solution}{Shengze Xu}
\chead{\hmwkClass\ : \hmwkTitle}
\rhead{\firstxmark}
\lfoot{\lastxmark}
\cfoot{\thepage}

\renewcommand\headrulewidth{0.4pt}
\renewcommand\footrulewidth{0.4pt}

\newcommand{\enterProblemHeader}[1]{
    \nobreak\extramarks{}{Problem \arabic{#1} continued on next page\ldots}\nobreak{}
    \nobreak\extramarks{Problem \arabic{#1} (continued)}{Problem \arabic{#1} continued on next page\ldots}\nobreak{}
}

\newcommand{\exitProblemHeader}[1]{
    \nobreak\extramarks{Problem \arabic{#1} (continued)}{Problem \arabic{#1} continued on next page\ldots}\nobreak{}
    \stepcounter{#1}
    \nobreak\extramarks{Problem \arabic{#1}}{}\nobreak{}
}

\setcounter{secnumdepth}{0}
\newcounter{partCounter}
\newcounter{homeworkProblemCounter}
\setcounter{homeworkProblemCounter}{1}
\nobreak\extramarks{Problem \arabic{homeworkProblemCounter}}{}\nobreak{}

\newenvironment{homeworkProblem}[1][-1]{
    \ifnum#1>0
        \setcounter{homeworkProblemCounter}{#1}
    \fi
    \section{Problem \arabic{homeworkProblemCounter}}
    \setcounter{partCounter}{1}
    \enterProblemHeader{homeworkProblemCounter}
}{
    \exitProblemHeader{homeworkProblemCounter}
}

\newcommand{\hmwkTitle}{HW12 Solution}
\newcommand{\hmwkDueDate}{\today}
\newcommand{\hmwkClass}{Scientific Computing}
\newcommand{\hmwkClassInstructor}{Professor Lai}
\newcommand{\hmwkAuthorName}{Xu Shengze 3190102721}

\title{
    \vspace{2in}
    \textmd{\textbf{\hmwkClass:\ \hmwkTitle}}\\
    \normalsize\vspace{0.1in}\small{\hmwkDueDate }\\
    \vspace{0.1in}\large{\textit{\hmwkClassInstructor\ }}
    \vspace{3in}
}

\author{\textbf{\hmwkAuthorName}}
\date{}

\renewcommand{\part}[1]{\textbf{\large Part \Alph{partCounter}}\stepcounter{partCounter}\\}

% Various Helper Commands
% Useful for algorithms
\newcommand{\alg}[1]{\textsc{\bfseries \footnotesize #1}}
% For derivatives
\newcommand{\deriv}[1]{\frac{\mathrm{d}}{\mathrm{d}x} (#1)}
% For partial derivatives
\newcommand{\pderiv}[2]{\frac{\partial}{\partial #1} (#2)}
% Integral dx
\newcommand{\dx}{\mathrm{d}x}
% Alias for the Solution section header
\newcommand{\solution}{\textbf{\large Solution}}
% Probability commands: Expectation, Variance, Covariance, Bias
\newcommand{\E}{\mathrm{E}}
\newcommand{\Var}{\mathrm{Var}}
\newcommand{\Cov}{\mathrm{Cov}}
\newcommand{\Bias}{\mathrm{Bias}}

\lstset{
	basicstyle=\tt,%行号
	numbers=left,
	rulesepcolor=\color{red!20!green!20!blue!20},
	escapeinside=``,
	xleftmargin=2em,xrightmargin=2em, aboveskip=1em,%背景框
	framexleftmargin=1.5mm,
	frame=shadowbox,%背景色
	backgroundcolor=\color[RGB]{245,245,244},%样式
	keywordstyle=\color{blue}\bfseries,
	identifierstyle=\bf,
	numberstyle=\color[RGB]{0,192,192},
	commentstyle=\it\color[RGB]{96,96,96},
	stringstyle=\rmfamily\slshape\color[RGB]{128,0,0},%显示空格
	showstringspaces=false
}

\begin{document}
\lstset{language=MATLAB}
\maketitle

\pagebreak

\begin{homeworkProblem}
Let $f(x)=3x^2-e^x$, then $f(x)$ is continuous on $R$. $f(0)=-1<0$, $f(1)=3-e>0$, so $3x^2-e^x=0$ has root on $[0,1]$, so we can take $a=0$ and $b=1$ to satisfy the problem.

From $f(x)=0$ we get $x=\sqrt{\frac{1}{3}e^x}=\varphi(x)$, so we do iterative $x_{n+1}=\varphi(x_n)$, i.e.$x_{n+1}=\sqrt{\frac{1}{3}e^{x_n}}$. Take the derivative of $\varphi(x)$, we get $\varphi'(x)=\frac{1}{2}\sqrt{\frac{1}{3}e^{x}}$. It's obvious that $\varphi'(x)$ is increasing on $[0,1]$, so $|\varphi'(x)|<\max{|\varphi'(0)|,|\varphi'(1)|}<1$, therefore the iterative method converges.
\end{homeworkProblem}

\begin{homeworkProblem}
Assume that $x_n=\sqrt{2+\sqrt{2+\cdots+\sqrt{2}}}$, than we know that $x_{n+1}=\sqrt{2+x_n}$ and $x_1=\sqrt{2}$. Take the derivative of $\sqrt{2+x}$, we can easily know that $|(\sqrt{2+x})'|=|\frac{1}{2\sqrt{2+x}}|<1$ when $x>0$, so the iteration coverges.

Assume that $\lim\limits_{n\to\infty}=\xi$, then we have $\xi=\sqrt{2+\xi}$, so $\xi=2$, i.e.$\lim\limits_{n\to\infty}\sqrt{2+\sqrt{2+\cdots+\sqrt{2}}}=2$.

\end{homeworkProblem}

\begin{homeworkProblem}
First, we write the function program of bisection method.
\begin{lstlisting}
function xc=bisection(f,a,b,eps)
	while (b-a)/2>eps
		c=(a+b)/2;     
		if f(c)==0        
			break 
		end
		if f(a)*f(c)<0   
			b=c;
		else                  
			a=c;
		end
	end
	xc=(a+b)/2;
end
\end{lstlisting}
Then we substitute the functional equation to solve the answer.
\end{homeworkProblem}
\begin{lstlisting}
format long e;
syms x;
f=inline("x-2^(-x)","x");
g=inline("exp(x)-x^2+3*x-2","x");
xc1=bisection(f,0,1,10^(-10));
xc2=bisection(g,0,1,10^(-10));
fprintf('解为%.11f\n',xc1);
fprintf('解为%.11f\n',xc2);
\end{lstlisting}
Run the program and get the following solution:
\begin{itemize}
\item[(1)]$x=0.6411857445$.

\item[(2)]$x=0.2575302855$.
\end{itemize}

\begin{homeworkProblem}
First, write the program according to the title.
\begin{lstlisting}
format long e;
x=1.5;
eps=1e-12;
N=10000; 
cnt=0;
while cnt<N
	x1=func(x);
	cnt=cnt+1;
	if abs(x1-x)<eps
		break;
	end
	x=x1;
end
fprintf('共迭代%d次,解为%.11f\n',cnt,x1);
\end{lstlisting}
The iterative function program used in the above program is as follows.
\begin{lstlisting}
function x = func(x)
	x=1+1/(x*x);
	%x=(1+x*x)^(1/3);
	%x=(1/(1-x))^(1/2);
end
\end{lstlisting}
According to the related mathematical theories and the results of program operation, we have the following conclusions.
\begin{itemize}
\item[(a)]Converges at $x_0=1.5$, a total of $56$ iterations, root is $x=1.4655712319$.
	
\item[(b)]Converges at $x_0=1.5$, a total of $32$ iterations, root is $x=1.4655712319$.

\item[(c)]Does not converge at $x_0=1.5$.
\end{itemize}
\end{homeworkProblem}
\end{document}
