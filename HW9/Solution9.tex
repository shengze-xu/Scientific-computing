\documentclass{article}

\usepackage{fancyhdr}
\usepackage[toc,page]{appendix}
\usepackage{extramarks}
\usepackage{amsmath}
\usepackage{amsthm}
\usepackage{amsfonts}
\usepackage{tikz}
\usepackage[plain]{algorithm}
\usepackage{algpseudocode}
\usepackage{amssymb}
\usepackage{bbm}
\usepackage{extarrows}
\usepackage{mathrsfs}
\usepackage{CJK}
\usepackage{dsfont}
\usepackage[hidelinks]{hyperref}
\usepackage{apacite}
\usepackage{multirow, booktabs}  
\usepackage{threeparttable}
\usepackage{dcolumn}
\usepackage{longtable}
\usepackage{threeparttablex}
\usepackage{tabu}
\usepackage{pdfpages}
\usepackage{float}
\usepackage{changepage}
\usepackage{mathtools}
\usepackage{listings}

\usetikzlibrary{automata,positioning}

\topmargin=-0.45in
\evensidemargin=0in
\oddsidemargin=0in
\textwidth=6.5in
\textheight=9.0in
\headsep=0.25in
\setlength{\parindent}{0em}
\linespread{1.1}

\pagestyle{myheadings}
\markboth{HW Solution}{Shengze Xu}
\chead{\hmwkClass\ : \hmwkTitle}
\rhead{\firstxmark}
\lfoot{\lastxmark}
\cfoot{\thepage}

\renewcommand\headrulewidth{0.4pt}
\renewcommand\footrulewidth{0.4pt}

\newcommand{\enterProblemHeader}[1]{
    \nobreak\extramarks{}{Problem \arabic{#1} continued on next page\ldots}\nobreak{}
    \nobreak\extramarks{Problem \arabic{#1} (continued)}{Problem \arabic{#1} continued on next page\ldots}\nobreak{}
}

\newcommand{\exitProblemHeader}[1]{
    \nobreak\extramarks{Problem \arabic{#1} (continued)}{Problem \arabic{#1} continued on next page\ldots}\nobreak{}
    \stepcounter{#1}
    \nobreak\extramarks{Problem \arabic{#1}}{}\nobreak{}
}

\setcounter{secnumdepth}{0}
\newcounter{partCounter}
\newcounter{homeworkProblemCounter}
\setcounter{homeworkProblemCounter}{1}
\nobreak\extramarks{Problem \arabic{homeworkProblemCounter}}{}\nobreak{}

\newenvironment{homeworkProblem}[1][-1]{
    \ifnum#1>0
        \setcounter{homeworkProblemCounter}{#1}
    \fi
    \section{Problem \arabic{homeworkProblemCounter}}
    \setcounter{partCounter}{1}
    \enterProblemHeader{homeworkProblemCounter}
}{
    \exitProblemHeader{homeworkProblemCounter}
}

\newcommand{\hmwkTitle}{HW9 Solution}
\newcommand{\hmwkDueDate}{\today}
\newcommand{\hmwkClass}{Scientific Computing}
\newcommand{\hmwkClassInstructor}{Professor Lai}
\newcommand{\hmwkAuthorName}{Xu Shengze 3190102721}

\title{
    \vspace{2in}
    \textmd{\textbf{\hmwkClass:\ \hmwkTitle}}\\
    \normalsize\vspace{0.1in}\small{\hmwkDueDate }\\
    \vspace{0.1in}\large{\textit{\hmwkClassInstructor\ }}
    \vspace{3in}
}

\author{\textbf{\hmwkAuthorName}}
\date{}

\renewcommand{\part}[1]{\textbf{\large Part \Alph{partCounter}}\stepcounter{partCounter}\\}

% Various Helper Commands
% Useful for algorithms
\newcommand{\alg}[1]{\textsc{\bfseries \footnotesize #1}}
% For derivatives
\newcommand{\deriv}[1]{\frac{\mathrm{d}}{\mathrm{d}x} (#1)}
% For partial derivatives
\newcommand{\pderiv}[2]{\frac{\partial}{\partial #1} (#2)}
% Integral dx
\newcommand{\dx}{\mathrm{d}x}
% Alias for the Solution section header
\newcommand{\solution}{\textbf{\large Solution}}
% Probability commands: Expectation, Variance, Covariance, Bias
\newcommand{\E}{\mathrm{E}}
\newcommand{\Var}{\mathrm{Var}}
\newcommand{\Cov}{\mathrm{Cov}}
\newcommand{\Bias}{\mathrm{Bias}}

\lstset{
	basicstyle=\tt,%行号
	numbers=left,
	rulesepcolor=\color{red!20!green!20!blue!20},
	escapeinside=``,
	xleftmargin=2em,xrightmargin=2em, aboveskip=1em,%背景框
	framexleftmargin=1.5mm,
	frame=shadowbox,%背景色
	backgroundcolor=\color[RGB]{245,245,244},%样式
	keywordstyle=\color{blue}\bfseries,
	identifierstyle=\bf,
	numberstyle=\color[RGB]{0,192,192},
	commentstyle=\it\color[RGB]{96,96,96},
	stringstyle=\rmfamily\slshape\color[RGB]{128,0,0},%显示空格
	showstringspaces=false
}

\begin{document}
\lstset{language=MATLAB}
\maketitle

\pagebreak

\begin{homeworkProblem}
The augmented matrix of the system of equations is as follows:
\begin{equation}
	\left[
	\begin{array}{ccc|c}
		1  & 2 & -1   & 1  \\
		-3 & 1 & 3   & 2 \\
		3  & -2 & 1   & 3 
	\end{array}
	\right]
\end{equation}
We perform the first transformation,
\begin{equation}
	\left[
	\begin{array}{ccc|c}
		3  & -2 & 1   & 3 \\
		-3 & 1 & 3   & 2 \\
		1  & 2 & -1   & 1 
	\end{array}
	\right]
\end{equation}
Next, we perform the second transformation,
\begin{equation}
	\left[
	\begin{array}{ccc|c}
		3  & -2 & 1   & 3 \\
		0  & -1 & 4   & 5 \\
		0  & \frac{8}{3} & -\frac{4}{3}   & 0 
	\end{array}
	\right]
\end{equation}
Then, we perform the third transformation,
\begin{equation}
	\left[
	\begin{array}{ccc|c}
		3  & -2 & 1   & 3 \\
		0  & \frac{8}{3} & -\frac{4}{3}   & 0 \\
		0  & -1 & 4   & 5 
	\end{array}
	\right]
\end{equation}
Then, we perform the last transformation,
\begin{equation}
	\left[
	\begin{array}{ccc|c}
		3  & -2 & 1   & 3 \\
		0  & \frac{8}{3} & -\frac{4}{3}   & 0 \\
		0  & 0 & \frac{7}{2}   &5 
	\end{array}
	\right]
\end{equation}

Now we can easily get the solution of the system of equations:
\begin{equation}
	\left\{
	\begin{aligned}
		x_1&=1\\
		x_2&=\frac{5}{7}\\
		x_3&=\frac{10}{7}
	\end{aligned}
	\right.
\end{equation}
\end{homeworkProblem}

\begin{homeworkProblem}
Assume that $A=(a_{ij})_{n\times n}$. According to the condition, $A$ is a matrix with a bandwidth of $2m+1$. $\forall |i-j|>m$, we have $a_{ij}=0$.

$A=LL^T$, $L=(l_{jk})_{n\times n}$.

We have the relation below:
\begin{equation}
	\left\{
	\begin{aligned}
		l_{kk}&=(a_{kk}-\sum_{i=1}^{k-1}l_{ki}^2)^{\frac{1}{2}}\\
		l_{jk}&=\frac{a_{jk}-\sum_{i=1}^{k-1}l_{ji}l_{ki}}{l_{kk}},  j=k+1,k+2,\cdots,n
	\end{aligned}
	\right.
\end{equation}
(i)When $k=1$, it's obvious that $l_{j1}=0$.

(ii)When $k=2$, $l_{j2}=\frac{a_{j2}-l_{j1}l_{21}}{l_{22}},j\geq m+3$.
$a_{j2}=0$ and $l_{j1}=0$, so $l_{j2}=0$.

(iii)Assume that the conclusion holds for $k$, consider the situation when $k+1$, at this time $l_{j,k+1}=\frac{a_{j,k+1}-\sum_{i=1}^{k}l_{ji}l_{k+1,i}}{l_{k+1,k+1}}$. $a_{j,k+1}=0$, $j-i>k+1+m-k=m+1>m$, $l_{ji}=0$, so $l_{j,k+1}=0$.

Therefore, $L$ is a matrix with a bandwidth of $m+1$.
\end{homeworkProblem}

\begin{homeworkProblem}
Frobenius norm of matrix $A$ is $||A||_F=(\sum_{i,j=1}^{n}|a_{ij}|^2)^{\frac{1}{2}}$.

(i)Positivity: $\forall A \in R^{n\times n}, ||A||_F=(\sum_{i,j=1}^{n}|a_{ij}|^2)^{\frac{1}{2}}\geq 0$, $||A||_F=0$ if and only if $\forall i,j,a_{ij}=0,i.e. A=0$.

(ii)Homogeneity: $\forall A \in R^{n\times n}$ and $\alpha \in R, ||\alpha A||_F=(\sum_{i,j=1}^{n}|\alpha a_{ij}|^2)^{\frac{1}{2}}=|\alpha|(\sum_{i,j=1}^{n}|a_{ij}|^2)^{\frac{1}{2}}=|\alpha|\cdot||A||_F$.

(iii)the Triangle Inequality: $\forall A,B \in R^{n\times n},||A+B||_F=(\sum_{i,j=1}^{n}|a_{ij}+b_{ij}|^2)^{\frac{1}{2}}\leq (\sum_{i,j=1}^{n}|a_{ij}|^2)^{\frac{1}{2}}+(\sum_{i,j=1}^{n}|b_{ij}|^2)^{\frac{1}{2}}=||A||_F+||B||_F$.

(iv)Compatibility: $\forall A,B \in R^{n\times n}, ||A\cdot B||=(\sum_{i,j=1}^{n}|a_{ij}b_{ij}|^2)^{\frac{1}{2}}\leq(\sum_{i,j=1}^{n}|a_{ij}|^2)^{\frac{1}{2}}(\sum_{i,j=1}^{n}|b_{ij}|^2)^{\frac{1}{2}}=||A||_F\cdot||B||_F$.

\end{homeworkProblem}

\begin{homeworkProblem}
\begin{itemize}
\item[(a)]
$||A||_F=(\sum_{i,j=1}^{n}|a_{ij}|^2)^{\frac{1}{2}}$, $||A||_2=max||Ax||_2=\sqrt{\lambda _1}$, 
take $x=(x_1,x_2,\cdots,x_n)^T$.

$||Ax||_2=||AX||_F\leq||A||_F||X||_F$, then $||Ax||_2\leq||A||_F$, so $||A||_2\leq||A||_F$.

$(||A||_F)^2=tr(A^TA)=\lambda _1+\lambda _2+\cdots+\lambda _n\leq n\lambda _1=n||A||_2^2$, so $||A||_F\leq\sqrt{n}||A||_2$.
\item[(b)]
$||x||_{\infty}\leq ||x||_{2}\leq \sqrt{n}||x||_{\infty}$.

Let $x'=Ax$, then $||Ax||_{2}\leq \sqrt{n}||Ax||_{\infty}$, so $||A||_{2}\leq \sqrt{n}||A||_{\infty}$.

$||A||_2=max\frac{||Ax||_2}{||x||_2}\geq \frac{\sqrt{\sum_{i=1}^{n}(\sum_{j=1}^{n}x_ja_{ij})^2}}{\sum_{j=1}^{n}x_j^2}$, take $x_{ij}=sgn(a_{ij})$, then $||A||_2\geq\frac{\sqrt{\sum_{i=1}^{n}(\sum_{j=1}^{n}|a_{ij}|)^2}}{\sqrt{n}}\geq \frac{max(\sum_{j=1}^{n}|a_{ij}|^2)}{\sqrt{n}}=\frac{1}{\sqrt{n}}||A||_{\infty}$.

\item[(c)]
$||x||_{1}\leq \sqrt{n}||x||_{2}$.

Let $x'=Ax$, then $||Ax||_{1}\leq \sqrt{n}||Ax||_{2}$, so $\frac{1}{\sqrt{n}}||A||_1\leq||A||_2$.

Meanwhile, $||x||_{\infty}\leq ||x||_{1}$, we can get $||A||_{\infty}\leq ||A||_{1}$, so $||A||_{2}\leq \sqrt{n}||A||_{\infty}\leq \sqrt{n}||A||_{1}$.

\end{itemize}
\end{homeworkProblem}


\begin{homeworkProblem}
First we generate the Hilbert matrix.
\begin{lstlisting}
function H=returnH(n)
	H=zeros(n,n);
	for i=1:n
		for j=1:n
			H(i,j)=1/(i+j-1);
		end
	end
end
\end{lstlisting}
Then, we use the column principal element Gaussian elimination method to solve fangchengzu.
\begin{lstlisting}
function [H,x1] = change(H,x1,n)
	for k=1:n-1
		a=H(k:n,k);
		[s,flag]=max(abs(a));
		if s==0
			break;
		end
		flag=flag+k-1;
		for j=1:n
			t=H(flag,j);
			H(flag,j)=H(k,j);
			H(k,j)=t;
		end 
		t=x1(flag);
		x1(flag)=x1(k);
		x1(k)=t;
		for i=k+1:n
			m(i,k)=H(i,k)/H(k,k);
			H(i,k)=m(i,k);
		end  
		for i=k+1:n
			for j=k+1:n
				H(i,j)=H(i,j)-m(i,k)*H(k,j);
			end
			x1(i)=x1(i)-m(i,k)*x1(k);
		end
	end
	for i=n:-1:1
		for j=i+1:n
			x1(i)=x1(i)-H(i,j)*x1(j);
		end
		x1(i)=x1(i)/H(i,i);
	end
end
\end{lstlisting}
According to the problem, set the loop condition to solve n.
\begin{lstlisting}
n=1;
m=zeros(n,n);
while 1
	H=returnH(n);H0=H;
	x=ones(n,1);b=H*x;x1=b; 
	[H,x1]=change(H,x1,n);
	if max(abs(x-x1))/max(abs(x))>=1
		break;
	end
	n=n+1;
end
\end{lstlisting}
By running the program, we got the following solution.

$n$: $13$.

$Cond$: $1.5192\times10^{16}$.

Infinite norm of backward error: $8.8818\times10^{-16}$.

Infinite norm of forward error: $4.2430$.


\end{homeworkProblem}
\end{document}
