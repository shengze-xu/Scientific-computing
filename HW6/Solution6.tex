\documentclass{article}

\usepackage{fancyhdr}
\usepackage[toc,page]{appendix}
\usepackage{extramarks}
\usepackage{amsmath}
\usepackage{amsthm}
\usepackage{amsfonts}
\usepackage{tikz}
\usepackage[plain]{algorithm}
\usepackage{algpseudocode}
\usepackage{amssymb}
\usepackage{bbm}
\usepackage{extarrows}
\usepackage{mathrsfs}
\usepackage{CJK}
\usepackage{dsfont}
\usepackage[hidelinks]{hyperref}
\usepackage{apacite}
\usepackage{multirow, booktabs}  
\usepackage{threeparttable}
\usepackage{dcolumn}
\usepackage{longtable}
\usepackage{threeparttablex}
\usepackage{tabu}
\usepackage{pdfpages}
\usepackage{float}
\usepackage{changepage}
\usepackage{mathtools}
\usepackage{listings}

\usetikzlibrary{automata,positioning}

\topmargin=-0.45in
\evensidemargin=0in
\oddsidemargin=0in
\textwidth=6.5in
\textheight=9.0in
\headsep=0.25in
\setlength{\parindent}{0em}
\linespread{1.1}

\pagestyle{myheadings}
\markboth{HW Solution}{Shengze Xu}
\chead{\hmwkClass\ : \hmwkTitle}
\rhead{\firstxmark}
\lfoot{\lastxmark}
\cfoot{\thepage}

\renewcommand\headrulewidth{0.4pt}
\renewcommand\footrulewidth{0.4pt}

\newcommand{\enterProblemHeader}[1]{
    \nobreak\extramarks{}{Problem \arabic{#1} continued on next page\ldots}\nobreak{}
    \nobreak\extramarks{Problem \arabic{#1} (continued)}{Problem \arabic{#1} continued on next page\ldots}\nobreak{}
}

\newcommand{\exitProblemHeader}[1]{
    \nobreak\extramarks{Problem \arabic{#1} (continued)}{Problem \arabic{#1} continued on next page\ldots}\nobreak{}
    \stepcounter{#1}
    \nobreak\extramarks{Problem \arabic{#1}}{}\nobreak{}
}

\setcounter{secnumdepth}{0}
\newcounter{partCounter}
\newcounter{homeworkProblemCounter}
\setcounter{homeworkProblemCounter}{1}
\nobreak\extramarks{Problem \arabic{homeworkProblemCounter}}{}\nobreak{}

\newenvironment{homeworkProblem}[1][-1]{
    \ifnum#1>0
        \setcounter{homeworkProblemCounter}{#1}
    \fi
    \section{Problem \arabic{homeworkProblemCounter}}
    \setcounter{partCounter}{1}
    \enterProblemHeader{homeworkProblemCounter}
}{
    \exitProblemHeader{homeworkProblemCounter}
}

\newcommand{\hmwkTitle}{HW6 Solution}
\newcommand{\hmwkDueDate}{\today}
\newcommand{\hmwkClass}{Scientific Computing}
\newcommand{\hmwkClassInstructor}{Professor Lai}
\newcommand{\hmwkAuthorName}{Xu Shengze 3190102721}

\title{
    \vspace{2in}
    \textmd{\textbf{\hmwkClass:\ \hmwkTitle}}\\
    \normalsize\vspace{0.1in}\small{\hmwkDueDate }\\
    \vspace{0.1in}\large{\textit{\hmwkClassInstructor\ }}
    \vspace{3in}
}

\author{\textbf{\hmwkAuthorName}}
\date{}

\renewcommand{\part}[1]{\textbf{\large Part \Alph{partCounter}}\stepcounter{partCounter}\\}

% Various Helper Commands
% Useful for algorithms
\newcommand{\alg}[1]{\textsc{\bfseries \footnotesize #1}}
% For derivatives
\newcommand{\deriv}[1]{\frac{\mathrm{d}}{\mathrm{d}x} (#1)}
% For partial derivatives
\newcommand{\pderiv}[2]{\frac{\partial}{\partial #1} (#2)}
% Integral dx
\newcommand{\dx}{\mathrm{d}x}
% Alias for the Solution section header
\newcommand{\solution}{\textbf{\large Solution}}
% Probability commands: Expectation, Variance, Covariance, Bias
\newcommand{\E}{\mathrm{E}}
\newcommand{\Var}{\mathrm{Var}}
\newcommand{\Cov}{\mathrm{Cov}}
\newcommand{\Bias}{\mathrm{Bias}}

\lstset{
	basicstyle=\tt,%行号
	numbers=left,
	rulesepcolor=\color{red!20!green!20!blue!20},
	escapeinside=``,
	xleftmargin=2em,xrightmargin=2em, aboveskip=1em,%背景框
	framexleftmargin=1.5mm,
	frame=shadowbox,%背景色
	backgroundcolor=\color[RGB]{245,245,244},%样式
	keywordstyle=\color{blue}\bfseries,
	identifierstyle=\bf,
	numberstyle=\color[RGB]{0,192,192},
	commentstyle=\it\color[RGB]{96,96,96},
	stringstyle=\rmfamily\slshape\color[RGB]{128,0,0},%显示空格
	showstringspaces=false
}

\begin{document}
\begin{CJK*}{GBK}{song}
\lstset{language=MATLAB}
\maketitle

\pagebreak

\begin{homeworkProblem}
For $T_n(f)$, divide $[a,b]$ into $n$ equal parts, we have the equation $T_n(f)=\frac{h}{2}(f(x_0)+2f(x_1)+...+2f(x_{n-1})+f(x_n))=\frac{h}{2}\sum_{k=0}^{n-1}(f(x_k)+f(x_{k+1}))$, where $h=\frac{b-a}{n}$.

$f(x)$ is continuous on $[a,b]$, so there is $\xi_k\in(x_k,x_{k+1})$, satisfying that $f(\xi_k)=\frac{f(x_k)+f(x_{k+1})}{2}$, then $T_n(f)=\sum_{k=0}^{n-1}h\cdot f(\xi_k)$.

According to the definition, we could easily know that $\lim\limits_{n\to\infty}T_n(f)=\lim\limits_{n\to\infty}\sum_{k=0}^{n-1}h\cdot f(\xi_k)=\int_{a}^{b}f(x)dx$.

For $S_n(f)$, divide $[a,b]$ into $2n$ equal parts, we also have $S_n(f)=\frac{h}{6}\sum_{k=1}^{n}(f(x_{2k-2})+4f(x_{2k-1})+f(x_{2k}))$.

$f(x)$ is continuous on $[a,b]$, so there is $\eta_k\in(x_{2k-2},x_{2k})$, satisfying that $f(\eta_k)=\frac{f(x_{2k-2})+4f(x_{2k-1})+f(x_{2k})}{6}$, then $S_n(f)=\sum_{k=1}^{n}h\cdot f(\eta_k)$.

$\lim\limits_{n\to\infty}S_n(f)=\lim\limits_{n\to\infty}\sum_{k=1}^{n}h\cdot f(\eta_k)=\int_{a}^{b}f(x)dx$.
\end{homeworkProblem}

\begin{homeworkProblem}
Divide $[a,b]$ into $2n$ equal parts, assume that $h$ is $\frac{b-a}{n}$.

According to $Problem1$, we already know that $S_n(f)=\frac{h}{6}\sum_{k=1}^{n}(f(x_{2k-2})+4f(x_{2k-1})+f(x_{2k}))$.

The same, $T_{2n}(f)=\frac{h}{4}\sum_{k=1}^{n}(f(x_{2k-2})+2f(x_{2k-1})+f(x_{2k})))$ and $T_n(f)=\frac{h}{2}\sum_{k=1}^{n}(f(x_{2k-2})+f(x_{2k}))$.

Therefore, according to the expression of $S_n(f)$, $T_n(f)$ and $T_{2n}(f)$, we have $\frac{4}{3}T_{2n}(f)-\frac{1}{3}T_n(f)=\frac{h}{6}\sum_{k=1}^{n}(2f(x_{2k-2})+4f(x_{2k-1})+2f(x_{2k})-f(x_{2k-2})-f(x_{2k}))=\frac{h}{6}\sum_{k=1}^{n}(f(x_{2k-2})+4f(x_{2k-1})+f(x_{2k}))=S_n(f)$.
\end{homeworkProblem}

\begin{homeworkProblem}
$T_n(x)=\cos(n\arccos{x})$.
\begin{itemize}
\item[(a)]
$T_{m+n}(x)+T_{m-n}(x)=\cos((m+n)\arccos{x})+\cos((m-n)\arccos{x})=2\cos(m\arccos{x})\cos(n\arccos{x})=2T_m(x)T_n(x)$.
\item[(b)]
$T_m(T_n(x))=\cos(m\arccos(\cos(n\arccos{x})))=\cos(mn\arccos{x})=T_{mn}(x)=\cos(n\arccos(\cos(m\arccos{x})))=T_n(T_m(x))$
\item[(c)]$T_{n+1}(x)=2xT_n(x)-T_{n-1}(x)$, $T_0(x)=1$, $T_1(x)=x$. 

Next, we use mathematical induction to prove the proposition.

Apparently, when $n=1$, $n=2$, the proposition holds. 

Assume that for $n\leq k$ the propositon holds.Then for $n=k+1$, $T_{k+1}(x)=2xT_{k}(x)-T_{k-1}(x)$, the degree of $T_{k+1}(x)$ is $k+1$, whose coefficient is $2^{k-1}\cdot 2=2^k$. Therefore, the proposition holds for $n=k+1$, inductive hypothesis holds.

In summary, the original proposition is established.
\end{itemize}
\end{homeworkProblem}

\begin{homeworkProblem}
Define $f(y)=MT_n(y)-P_n(y)$. For $M=max|P(x)|$ on $[-1,1]$, we have $-M\leq P_n(y)\leq M$. 

According to the properties of Chebyshev polynomials, we know that $-M\leq MT_n(y)\leq M$ and there is $n+1$ values let $MT_n(y)$ take $-M$ or $M$. 

Therefore, $MT_n(y)$ and $P_n(y)$ have $n$ intersections, then $f(y)$ has $n$ zero points on $[-1,1]$. Meanwhile, the degree of $f(y)$ is no more than $n$, so there are no more than $n$ zero points. 

Obviously, there is no zero point when $y\textgreater 1$ or $y\textless -1$. $T_n(1)=1$, so $f(1)=M-P_n(y)\geq 0$, then when $y\geq 1$, $f(1)\geq 0$. Therefore, we have $P_n(y)\leq MT_n(y)$.

The same, if we define $g(y)=MT_n(y)+P_n(y)$, we can also get the conclusion that $g(1)\geq 0$, then $P_n(y)\geq -MT_n(y)$.

In summary, we have the conclusion that $|P_n(y)|\leq M|T_n(y)|$ for every $y\textgreater 1$.
\end{homeworkProblem}

\begin{homeworkProblem}
First, we define the function $comtrapezium(f,n,a,b)$.
\begin{lstlisting}
function ans=comtrapezium(f,n,a,b)
	h=(b-a)/n;
	sum=f(a);
	for i=a+h:h:b-h 
		sum=sum+2*f(i);
	end
	sum=sum+f(b);
	ans=sum*h/2;
end
\end{lstlisting}
Then, we should determine the value of $n$ and get the answer.
\begin{lstlisting}
format long e
syms x1;
f1=exp(x1)*sin(x1);
ans=1.0/(1*10^(-6))/12*double(max(abs(subs(diff(f1,x1,2),x1,[1:0.001:2]))));
n=ceil(sqrt(ans));
f=inline('exp(x)*sin(x)','x');
ans1=comtrapezium(f,n,1,2);
true=int(f1,x1,1,2);
disp(n);
disp(ans1);
disp(vpa(true));
\end{lstlisting}
According to the error formula, we get the value of $n$ is $716$.

By running the program, the calculated result we get is $4.487560317$, the true value is $4.487560335$.
\end{homeworkProblem}

\begin{homeworkProblem}
First, we define the function $comsimpson(f,n,a,b)$.
\begin{lstlisting}
function ans=comsimpson(f,n,a,b)
	format long;
	h=(b-a)/n;
	sum=f(a);
	for i=a+h:h:b-h 
		sum=sum+2*f(i);
	end
	for i=a+h/2:h:b-h/2 
		sum=sum+4*f(i);
	end
	sum=sum+f(b);
	ans=sum*h/6;
end
\end{lstlisting}
Next, we get the value of $n$ according to the formula and 	get the answer by calculation.
\begin{lstlisting}
format long e
f=inline('exp(x)*sin(x)','x');
n=1;
ans1=comsimpson(f,n,1,3);
ans2=comsimpson(f,2*n,1,3);
while abs(ans1-ans2)>10^(-8)
	n=n+1;
	ans1=comsimpson(f,n,1,3);
	ans2=comsimpson(f,2*n,1,3);
end
disp(ans1);
disp(ans2);
\end{lstlisting}
\end{homeworkProblem}
By running the program, the calculated result we get is $10.95017031$.
\end{CJK*}
\end{document}
