\documentclass{article}

\usepackage{fancyhdr}
\usepackage[toc,page]{appendix}
\usepackage{extramarks}
\usepackage{amsmath}
\usepackage{amsthm}
\usepackage{amsfonts}
\usepackage{tikz}
\usepackage[plain]{algorithm}
\usepackage{algpseudocode}
\usepackage{amssymb}
\usepackage{bbm}
\usepackage{extarrows}
\usepackage{mathrsfs}
\usepackage{CJK}
\usepackage{dsfont}
\usepackage[hidelinks]{hyperref}
\usepackage{apacite}
\usepackage{multirow, booktabs}  
\usepackage{threeparttable}
\usepackage{dcolumn}
\usepackage{longtable}
\usepackage{threeparttablex}
\usepackage{tabu}
\usepackage{pdfpages}
\usepackage{float}
\usepackage{changepage}
\usepackage{mathtools}
\usepackage{listings}

\usetikzlibrary{automata,positioning}

\topmargin=-0.45in
\evensidemargin=0in
\oddsidemargin=0in
\textwidth=6.5in
\textheight=9.0in
\headsep=0.25in
\setlength{\parindent}{0em}
\linespread{1.1}

\pagestyle{myheadings}
\markboth{HW Solution}{Shengze Xu}
\chead{\hmwkClass\ : \hmwkTitle}
\rhead{\firstxmark}
\lfoot{\lastxmark}
\cfoot{\thepage}

\renewcommand\headrulewidth{0.4pt}
\renewcommand\footrulewidth{0.4pt}

\newcommand{\enterProblemHeader}[1]{
    \nobreak\extramarks{}{Problem \arabic{#1} continued on next page\ldots}\nobreak{}
    \nobreak\extramarks{Problem \arabic{#1} (continued)}{Problem \arabic{#1} continued on next page\ldots}\nobreak{}
}

\newcommand{\exitProblemHeader}[1]{
    \nobreak\extramarks{Problem \arabic{#1} (continued)}{Problem \arabic{#1} continued on next page\ldots}\nobreak{}
    \stepcounter{#1}
    \nobreak\extramarks{Problem \arabic{#1}}{}\nobreak{}
}

\setcounter{secnumdepth}{0}
\newcounter{partCounter}
\newcounter{homeworkProblemCounter}
\setcounter{homeworkProblemCounter}{1}
\nobreak\extramarks{Problem \arabic{homeworkProblemCounter}}{}\nobreak{}

\newenvironment{homeworkProblem}[1][-1]{
    \ifnum#1>0
        \setcounter{homeworkProblemCounter}{#1}
    \fi
    \section{Problem \arabic{homeworkProblemCounter}}
    \setcounter{partCounter}{1}
    \enterProblemHeader{homeworkProblemCounter}
}{
    \exitProblemHeader{homeworkProblemCounter}
}

\newcommand{\hmwkTitle}{HW8 Solution}
\newcommand{\hmwkDueDate}{\today}
\newcommand{\hmwkClass}{Scientific Computing}
\newcommand{\hmwkClassInstructor}{Professor Lai}
\newcommand{\hmwkAuthorName}{Xu Shengze 3190102721}

\title{
    \vspace{2in}
    \textmd{\textbf{\hmwkClass:\ \hmwkTitle}}\\
    \normalsize\vspace{0.1in}\small{\hmwkDueDate }\\
    \vspace{0.1in}\large{\textit{\hmwkClassInstructor\ }}
    \vspace{3in}
}

\author{\textbf{\hmwkAuthorName}}
\date{}

\renewcommand{\part}[1]{\textbf{\large Part \Alph{partCounter}}\stepcounter{partCounter}\\}

% Various Helper Commands
% Useful for algorithms
\newcommand{\alg}[1]{\textsc{\bfseries \footnotesize #1}}
% For derivatives
\newcommand{\deriv}[1]{\frac{\mathrm{d}}{\mathrm{d}x} (#1)}
% For partial derivatives
\newcommand{\pderiv}[2]{\frac{\partial}{\partial #1} (#2)}
% Integral dx
\newcommand{\dx}{\mathrm{d}x}
% Alias for the Solution section header
\newcommand{\solution}{\textbf{\large Solution}}
% Probability commands: Expectation, Variance, Covariance, Bias
\newcommand{\E}{\mathrm{E}}
\newcommand{\Var}{\mathrm{Var}}
\newcommand{\Cov}{\mathrm{Cov}}
\newcommand{\Bias}{\mathrm{Bias}}

\lstset{
	basicstyle=\tt,%行号
	numbers=left,
	rulesepcolor=\color{red!20!green!20!blue!20},
	escapeinside=``,
	xleftmargin=2em,xrightmargin=2em, aboveskip=1em,%背景框
	framexleftmargin=1.5mm,
	frame=shadowbox,%背景色
	backgroundcolor=\color[RGB]{245,245,244},%样式
	keywordstyle=\color{blue}\bfseries,
	identifierstyle=\bf,
	numberstyle=\color[RGB]{0,192,192},
	commentstyle=\it\color[RGB]{96,96,96},
	stringstyle=\rmfamily\slshape\color[RGB]{128,0,0},%显示空格
	showstringspaces=false
}

\begin{document}
\lstset{language=MATLAB}
\maketitle

\pagebreak

\begin{homeworkProblem}
The augmented matrix of the system of equations is as follows:
\begin{equation}
\left[
\begin{array}{cccc|c}
	6  & 2 & 1  & -1 & 6  \\
	2  & 4 & 1  & 0  & -1 \\
	1  & 1 & 4  & -1 & 5  \\
	-1 & 0 & -1 & 3  & -5
\end{array}
\right]
\end{equation}

We perform the first transformation,
\begin{equation}
	\left[
	\begin{array}{cccc|c}
		6  & 2 & 1  & -1 & 6  \\
		0  & \frac{10}{3} & \frac{2}{3}  & \frac{1}{3}  & -3 \\
		0  & \frac{2}{3} & \frac{23}{6}  & -\frac{5}{6} & 4  \\
		0 & \frac{1}{3} & -\frac{5}{6} & \frac{17}{6}  & -4
	\end{array}
	\right]
\end{equation}

Next, we perform the second transformation,
\begin{equation}
	\left[
	\begin{array}{cccc|c}
		6  & 2 & 1  & -1 & 6  \\
		0  & \frac{10}{3} & \frac{2}{3}  & \frac{1}{3}  & -3 \\
		0  & 0 & \frac{37}{10}  & -\frac{9}{10} &  \frac{23}{5} \\
		0 & 0 & -\frac{9}{10} & \frac{14}{5}  & -\frac{37}{10}
	\end{array}
	\right]
\end{equation}

Then, we perform the third transformation,
\begin{equation}
	\left[
	\begin{array}{cccc|c}
		6  & 2 & 1  & -1 & 6  \\
		0  & \frac{10}{3} & \frac{2}{3}  & \frac{1}{3}  & -3 \\
		0  & 0 & \frac{37}{10}  & -\frac{9}{10} &  \frac{23}{5} \\
		0 & 0 & 0 & \frac{191}{74}  & -\frac{191}{74}
	\end{array}
	\right]
\end{equation}

Now we can easily get the solution of the system of equations:
\begin{equation}
	\left\{
	\begin{aligned}
		x_1&=1\\
		x_2&=-1\\
		x_3&=1\\
		x_4&=-1
	\end{aligned}
	\right.
\end{equation}
\end{homeworkProblem}

\begin{homeworkProblem}
We write $A$ as follow:
\begin{equation}
	A=\left[\begin{array}{cccc}
		a_{11} & a_{12} & \cdots & a_{1n} \\
	    &a_{22}  & \cdots &a_{2n}  \\
		&  & \ddots & \vdots \\
		& &  &a_{nn}
	\end{array}\right]
\end{equation}
Assume that $B$ is the inverse of $A$ and we define $B$ like $A$ above, then we have:
\begin{equation}
	\left[\begin{array}{cccc}
		a_{11} & a_{12} & \cdots & a_{1n} \\
		&a_{22}  & \cdots &a_{2n}  \\
		&  & \ddots & \vdots \\
		& &  &a_{nn}
	\end{array}\right]
	\left[\begin{array}{cccc}
		b_{11} & b_{12} & \cdots & b_{1n} \\
		&b_{22}  & \cdots &b_{2n}  \\
		&  & \ddots & \vdots \\
		& &  &b_{nn}
	\end{array}\right]
	=I
\end{equation}

We get the relation:
\begin{equation}
	\sum_{k=i}^{n}a_{ik}b_{kj}=c_{ij}=\begin{cases}
		1&,i=j\\
		0&,i\neq j
	\end{cases}
\end{equation}
Firstly, by observing the last line we know $b_{nn}=\frac{1}{a_{nn}}$, further we could easily get $b_{ii}=\frac{1}{a_{ii}}$.

Next we consider the penultimate line and get the relation $a_{n-1,n-1}b_{n-1,n}+a_{n-1,n}b_{n,n}=0$, then we know $b_{n-1,n}=-\frac{a_{n-1,n}}{a_{n-1,n-1}a_{n,n}}$.

For the $l^{th}$ line from the bottom, we have $\sum_{k=l}^{n}a_{lk}b_{kj}=c_{lj}$. When $j\geq l+1$, we have $\sum_{k=l}^{n}a_{lk}b_{kj}=0$, then $a_{ll}b_{lj}+\sum_{k=l+1}^{n}a_{lk}b_{kj}=0$, so $b_{lj}=-\frac{\sum_{k=l+1}^{n}a_{lk}b_{kj}}{a_{ll}}$ for the other elements are known quantities.

Replace subscript, we get the recurrence expression, $b_{ii}=\frac{1}{a_{ii}}$ and $b_{ij}=-\frac{\sum_{k=i+1}^{n}a_{ik}b_{kj}}{a_{ii}}(j>i)$. 

What's interesting is that, unlike usual, this recursive expression is pushed from back to front.
\end{homeworkProblem}

\begin{homeworkProblem}
Assume that the LU decomposition of A is not unique, then we can write A as $A=L_1U_1=L_2U_2$.

Because $L_i,U_i(i=1,2)$ are upper triangular matrices, they have inverse matrix, and $L_2^{-1}L_1=U_1^{-1}U_2$.

Meanwhile, the inverse of the upper triangular matrix is still the upper triangular matrix, and the inverse of the lower triangular matrix is still the lower triangular matrix.

Therefore, $L_2^{-1}L_1=U_1^{-1}U_2=E$, then we have $L_1=L_2,U_1=U_2$, the proposition is proved.

\end{homeworkProblem}

\begin{homeworkProblem}
Assume that $A=LL^{T}$ and write the code according to the formula.
\begin{lstlisting}
A=zeros(10,10);
A(1,1)=9;A(1,2)=-4;A(1,3)=1;
A(2,1)=-4;A(2,2)=6;A(2,3)=-4;A(2,4)=1;
for i=3:10-2
	A(i,i-2)=1;A(i,i-1)=-4;A(i,i)=6;A(i,i+1)=-4;A(i,i+2)=1;
end 
A(9,7)=1;A(9,8)=-4;A(9,9)=5;A(9,10)=-2;
A(10,8)=1;A(10,9)=-2 ;A(10,10)=1;
disp(A);

L=zeros(10,10);
for i=10:-1:1
	L(i,i)=A(i,i);
	for k=i+1:1:10
		L(i,i)=L(i,i)-L(i,k)*L(i,k);
	end
	L(i,i)=sqrt(L(i,i));
	for j=1:1:i-1
		L(j,i)=A(i,j);
		for k=i+1:1:10
			L(j,i)=L(j,i)-L(i,k)*L(j,k);
		end
	L(j,i)=L(j,i)*1.0/L(i,i);
	end
end
disp(L);
\end{lstlisting}

By running the program, we get the matrix L:
\begin{equation}
	L=\left[
	\begin{array}{cccccccccc}
		2 & -2 & 1  & 0  & 0  & 0  & 0  & 0  & 0  & 0  \\
		0 & 1  & -2 & 1  & 0  & 0  & 0  & 0  & 0  & 0  \\
		0 & 0  & 1  & -2 & 1  & 0  & 0  & 0  & 0  & 0  \\
		0 & 0  & 0  & 1  & -2 & 1  & 0  & 0  & 0  & 0  \\
		0 & 0  & 0  & 0  & 1  & -2 & 1  & 0  & 0  & 0  \\
		0 & 0  & 0  & 0  & 0  & 1  & -2 & 1  & 0  & 0  \\
		0 & 0  & 0  & 0  & 0  & 0  & 1  & -2 & 1  & 0  \\
		0 & 0  & 0  & 0  & 0  & 0  & 0  & 1  & -2 & 1  \\
		0 & 0  & 0  & 0  & 0  & 0  & 0  & 0  & 1  & -2 \\
		0 & 0  & 0  & 0  & 0  & 0  & 0  & 0  & 0  & 1 
	\end{array}
	\right]
\end{equation}
\end{homeworkProblem}
\end{document}
