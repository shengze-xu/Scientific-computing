\documentclass{article}

\usepackage{fancyhdr}
\usepackage[toc,page]{appendix}
\usepackage{extramarks}
\usepackage{amsmath}
\usepackage{amsthm}
\usepackage{amsfonts}
\usepackage{tikz}
\usepackage[plain]{algorithm}
\usepackage{algpseudocode}
\usepackage{amssymb}
\usepackage{bbm}
\usepackage{extarrows}
\usepackage{mathrsfs}
\usepackage{CJK}
\usepackage{dsfont}
\usepackage[hidelinks]{hyperref}
\usepackage{apacite}
\usepackage{multirow, booktabs}  
\usepackage{threeparttable}
\usepackage{dcolumn}
\usepackage{longtable}
\usepackage{threeparttablex}
\usepackage{tabu}
\usepackage{pdfpages}
\usepackage{float}
\usepackage{changepage}
\usepackage{mathtools}
\usepackage{listings}
\lstset{language=Matlab}
\usetikzlibrary{automata,positioning}

\topmargin=-0.45in
\evensidemargin=0in
\oddsidemargin=0in
\textwidth=6.5in
\textheight=9.0in
\headsep=0.25in
\setlength{\parindent}{0em}
\linespread{1.1}

\pagestyle{myheadings}
\markboth{HW Solution}{Shengze Xu}
\chead{\hmwkClass\ : \hmwkTitle}
\rhead{\firstxmark}
\lfoot{\lastxmark}
\cfoot{\thepage}

\renewcommand\headrulewidth{0.4pt}
\renewcommand\footrulewidth{0.4pt}

\newcommand{\enterProblemHeader}[1]{
    \nobreak\extramarks{}{Problem \arabic{#1} continued on next page\ldots}\nobreak{}
    \nobreak\extramarks{Problem \arabic{#1} (continued)}{Problem \arabic{#1} continued on next page\ldots}\nobreak{}
}

\newcommand{\exitProblemHeader}[1]{
    \nobreak\extramarks{Problem \arabic{#1} (continued)}{Problem \arabic{#1} continued on next page\ldots}\nobreak{}
    \stepcounter{#1}
    \nobreak\extramarks{Problem \arabic{#1}}{}\nobreak{}
}

\setcounter{secnumdepth}{0}
\newcounter{partCounter}
\newcounter{homeworkProblemCounter}
\setcounter{homeworkProblemCounter}{1}
\nobreak\extramarks{Problem \arabic{homeworkProblemCounter}}{}\nobreak{}

\newenvironment{homeworkProblem}[1][-1]{
    \ifnum#1>0
        \setcounter{homeworkProblemCounter}{#1}
    \fi
    \section{Problem \arabic{homeworkProblemCounter}}
    \setcounter{partCounter}{1}
    \enterProblemHeader{homeworkProblemCounter}
}{
    \exitProblemHeader{homeworkProblemCounter}
}

\newcommand{\hmwkTitle}{HW4 Solution}
\newcommand{\hmwkDueDate}{\today}
\newcommand{\hmwkClass}{Scientific Computing}
\newcommand{\hmwkClassInstructor}{Professor Lai}
\newcommand{\hmwkAuthorName}{Xu Shengze 3190102721}

\title{
    \vspace{2in}
    \textmd{\textbf{\hmwkClass:\ \hmwkTitle}}\\
    \normalsize\vspace{0.1in}\small{\hmwkDueDate }\\
    \vspace{0.1in}\large{\textit{\hmwkClassInstructor\ }}
    \vspace{3in}
}

\author{\textbf{\hmwkAuthorName}}
\date{}

\renewcommand{\part}[1]{\textbf{\large Part \Alph{partCounter}}\stepcounter{partCounter}\\}

% Various Helper Commands
% Useful for algorithms
\newcommand{\alg}[1]{\textsc{\bfseries \footnotesize #1}}
% For derivatives
\newcommand{\deriv}[1]{\frac{\mathrm{d}}{\mathrm{d}x} (#1)}
% For partial derivatives
\newcommand{\pderiv}[2]{\frac{\partial}{\partial #1} (#2)}
% Integral dx
\newcommand{\dx}{\mathrm{d}x}
% Alias for the Solution section header
\newcommand{\solution}{\textbf{\large Solution}}
% Probability commands: Expectation, Variance, Covariance, Bias
\newcommand{\E}{\mathrm{E}}
\newcommand{\Var}{\mathrm{Var}}
\newcommand{\Cov}{\mathrm{Cov}}
\newcommand{\Bias}{\mathrm{Bias}}

\lstset{
	basicstyle=\tt,%行号
	numbers=left,
	rulesepcolor=\color{red!20!green!20!blue!20},
	escapeinside=``,
	xleftmargin=2em,xrightmargin=2em, aboveskip=1em,%背景框
	framexleftmargin=1.5mm,
	frame=shadowbox,%背景色
	backgroundcolor=\color[RGB]{245,245,244},%样式
	keywordstyle=\color{blue}\bfseries,
	identifierstyle=\bf,
	numberstyle=\color[RGB]{0,192,192},
	commentstyle=\it\color[RGB]{96,96,96},
	stringstyle=\rmfamily\slshape\color[RGB]{128,0,0},%显示空格
	showstringspaces=false
}

\begin{document}
\begin{CJK*}{GBK}{song}
\lstset{language=C}
\maketitle

\pagebreak

\begin{homeworkProblem}
Assume that $x_1$,$x_2$ are two least square solutions of $Ax=b$, therefore we have $A^TAx_1=A^Tb=A^TAx_2$.

From the equation above, we have $A^TA(x_1-x_2)=0$, so $(x_1-x_2)^TA^TA(x_1-x_2)=0$, i.e.$(A(x_1-x_2))^T(A(x_1-x_2))=0$, then $||A(x_1-x_2)||=0$, so $A(x_1-x_2)=0$.

Therefore, we have $Ax_1=Ax_2$.
\end{homeworkProblem}

\begin{homeworkProblem}
$\sum_{i=1}^{r}||Ax-b_i||^2=\sum_{i=1}^{r}(Ax-b_i)^T(Ax-b_i)\\=\sum_{i=1}^{r}(x^TA^TAx+b_i^Tb_i-x^TA^Tb_i-b_i^TAx)\\=r(||Ax||^2-(Ax)^T(\frac{1}{r}\sum_{i=1}^{r}b_i)-(\frac{1}{r}\sum_{i=1}^{r}b_i)^T(Ax))+\sum_{i=1}^{r}||b_i||^2\\=r||Ax-\frac{1}{r}\sum_{i=1}^{r}b_i||^2+\sum_{i=1}^{r}||b_i||^2-\frac{1}{r}||\sum_{i=1}^{r}b_i||^2$

Therefore, when $\sum_{i=1}^{r}||Ax-b_i||^2$ take minimum, $x$ is also the least square solution of $Ax=\frac{1}{r}\sum_{i=1}^{r}b_i$, vice versa.
\end{homeworkProblem}

\begin{homeworkProblem}
Assume that the $m\times{m}$ coefficient matrix is $X$, where $X_{i,j}=x_i^{j-1}, i,j={1,2,...,m}$. 

$\alpha=[a_0,a_1,...,a_{m-1}]^T$, $y=[y_1,y_2,...,y_m]^T$.

It's easy to find that $|X|\neq0$, so $X$ is invertible and $X\alpha=y$ has the unique solution $\alpha=X^{-1}y$.

For Lagrange interpolation polynomial $f(x)$, its degree is $m-1$.

Let $g(x)=p(x)-f(x)$, so $g(x_i)=p(x_i)-f(x_i)=0$. It's obvious that the degree of $g(x)\leq{m-1}$. Meanwhile, $g(x)$ has $m$ roots, therefore $g(x)$ must be $0$, further, we have $p(x)=f(x)$.
\end{homeworkProblem}

\begin{homeworkProblem}
\begin{lstlisting}
format rat
x=[-3;-2;-1;0;1;2;3];
y=[4;2;3;0;-1;-2;-5];
X=ones(7,3);
X(:,2)=x;
X(:,3)=x.^2;
a=(X'*X)\(X'*y);
disp(a);
\end{lstlisting}
According to the program, we have $a_0=\frac{2}{3}=0.667,a_1=-\frac{39}{28}=-1.393,a_2=-\frac{11}{84}=-0.131$.
\end{homeworkProblem}

\begin{homeworkProblem}
\begin{lstlisting}
format rat
x=[1.02;0.95;0.87;0.77;0.67;0.56;0.44;0.30;0.16;0.01];
y=[0.39;0.32;0.27;0.22;0.18;0.15;0.13;0.12;0.13;0.15];
X=ones(10,5);
X(:,1)=y.^2;
X(:,2)=x.*y;
X(:,3)=x;
X(:,4)=y;
a=(X'*X)\(X'*(x.^2));
disp(a);
\end{lstlisting}
According to the program, we have $a=-\frac{651}{247}=-2.636,b=\frac{3511}{24442}=0.144,c=\frac{343}{622}=0.551,d=\frac{1995}{619}=3.223,e=-\frac{1216}{2809}=-0.433$.
\end{homeworkProblem}
\end{CJK*}
\end{document}
