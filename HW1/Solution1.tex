\documentclass{article}

\usepackage{fancyhdr}
\usepackage[toc,page]{appendix}
\usepackage{extramarks}
\usepackage{amsmath}
\usepackage{amsthm}
\usepackage{amsfonts}
\usepackage{tikz}
\usepackage[plain]{algorithm}
\usepackage{algpseudocode}
\usepackage{amssymb}
\usepackage{bbm}
\usepackage{extarrows}
\usepackage{mathrsfs}
\usepackage{CJK}
\usepackage{dsfont}
\usepackage[hidelinks]{hyperref}
\usepackage{apacite}
\usepackage{multirow, booktabs}  
\usepackage{threeparttable}
\usepackage{dcolumn}
\usepackage{longtable}
\usepackage{threeparttablex}
\usepackage{tabu}
\usepackage{pdfpages}
\usepackage{float}
\usepackage{changepage}
\usepackage{mathtools}
\usepackage{listings}

\usetikzlibrary{automata,positioning}

\topmargin=-0.45in
\evensidemargin=0in
\oddsidemargin=0in
\textwidth=6.5in
\textheight=9.0in
\headsep=0.25in
\setlength{\parindent}{0em}
\linespread{1.1}

\pagestyle{myheadings}
\markboth{HW Solution}{Shengze Xu}
\chead{\hmwkClass\ : \hmwkTitle}
\rhead{\firstxmark}
\lfoot{\lastxmark}
\cfoot{\thepage}

\renewcommand\headrulewidth{0.4pt}
\renewcommand\footrulewidth{0.4pt}

\newcommand{\enterProblemHeader}[1]{
    \nobreak\extramarks{}{Problem \arabic{#1} continued on next page\ldots}\nobreak{}
    \nobreak\extramarks{Problem \arabic{#1} (continued)}{Problem \arabic{#1} continued on next page\ldots}\nobreak{}
}

\newcommand{\exitProblemHeader}[1]{
    \nobreak\extramarks{Problem \arabic{#1} (continued)}{Problem \arabic{#1} continued on next page\ldots}\nobreak{}
    \stepcounter{#1}
    \nobreak\extramarks{Problem \arabic{#1}}{}\nobreak{}
}

\setcounter{secnumdepth}{0}
\newcounter{partCounter}
\newcounter{homeworkProblemCounter}
\setcounter{homeworkProblemCounter}{1}
\nobreak\extramarks{Problem \arabic{homeworkProblemCounter}}{}\nobreak{}

\newenvironment{homeworkProblem}[1][-1]{
    \ifnum#1>0
        \setcounter{homeworkProblemCounter}{#1}
    \fi
    \section{Problem \arabic{homeworkProblemCounter}}
    \setcounter{partCounter}{1}
    \enterProblemHeader{homeworkProblemCounter}
}{
    \exitProblemHeader{homeworkProblemCounter}
}

\newcommand{\hmwkTitle}{HW1 Solution}
\newcommand{\hmwkDueDate}{\today}
\newcommand{\hmwkClass}{Scientific Computing}
\newcommand{\hmwkClassInstructor}{Professor Lai}
\newcommand{\hmwkAuthorName}{Xu Shengze 3190102721}

\title{
    \vspace{2in}
    \textmd{\textbf{\hmwkClass:\ \hmwkTitle}}\\
    \normalsize\vspace{0.1in}\small{\hmwkDueDate }\\
    \vspace{0.1in}\large{\textit{\hmwkClassInstructor\ }}
    \vspace{3in}
}

\author{\textbf{\hmwkAuthorName}}
\date{}

\renewcommand{\part}[1]{\textbf{\large Part \Alph{partCounter}}\stepcounter{partCounter}\\}

% Various Helper Commands
% Useful for algorithms
\newcommand{\alg}[1]{\textsc{\bfseries \footnotesize #1}}
% For derivatives
\newcommand{\deriv}[1]{\frac{\mathrm{d}}{\mathrm{d}x} (#1)}
% For partial derivatives
\newcommand{\pderiv}[2]{\frac{\partial}{\partial #1} (#2)}
% Integral dx
\newcommand{\dx}{\mathrm{d}x}
% Alias for the Solution section header
\newcommand{\solution}{\textbf{\large Solution}}
% Probability commands: Expectation, Variance, Covariance, Bias
\newcommand{\E}{\mathrm{E}}
\newcommand{\Var}{\mathrm{Var}}
\newcommand{\Cov}{\mathrm{Cov}}
\newcommand{\Bias}{\mathrm{Bias}}

\lstset{
	basicstyle=\tt,%行号
	numbers=left,
	rulesepcolor=\color{red!20!green!20!blue!20},
	escapeinside=``,
	xleftmargin=2em,xrightmargin=2em, aboveskip=1em,%背景框
	framexleftmargin=1.5mm,
	frame=shadowbox,%背景色
	backgroundcolor=\color[RGB]{245,245,244},%样式
	keywordstyle=\color{blue}\bfseries,
	identifierstyle=\bf,
	numberstyle=\color[RGB]{0,192,192},
	commentstyle=\it\color[RGB]{96,96,96},
	stringstyle=\rmfamily\slshape\color[RGB]{128,0,0},%显示空格
	showstringspaces=false
}

\begin{document}
\begin{CJK*}{GBK}{song}
\lstset{language=C}
\maketitle

\pagebreak

\begin{homeworkProblem}
\begin{itemize}
\item[(1)]bounds of absolute error:$\frac{1}{2}\times10^{-4}$ 

bounds of relative error: $\frac{1}{8}\times10^{-2}$ 

number of significant figures:3
\item[(2)]bounds of absolute error:$\frac{1}{2}\times10^{-4}$ 

bounds of relative error: $\frac{1}{8}\times10^{-3}$ 

number of significant figures:4
\item[(3)]bounds of absolute error:$\frac{1}{2}\times10^{-2}$ 

bounds of relative error: $\frac{1}{6}\times10^{-3}$ 

number of significant figures:4
\item[(4)]bounds of absolute error: $\frac{1}{2}$ 

bounds of relative error: $\frac{1}{8}\times10^{-3}$ 

number of significant figures:4
\end{itemize}
\end{homeworkProblem}

\begin{homeworkProblem}
\begin{itemize}
\item[(1)]$\tan(\arctan(x+1)-\arctan(x))=\frac{(x+1)-(x)}{1+x(x+1)}=\frac{1}{1+x+x^{2}}$

therefore,we have $\arctan(x+1)-\arctan(x)=\arctan(\frac{1}{1+x+x^{2}})$
\item[(2)]$\ln(x-\sqrt{x^{2}-1})=\frac{1}{ln(x+\sqrt{x^{2}-1})}$
\item[(3)]$\frac{\sin{x}}{x-\sqrt{x^{2}-1}}=\sin{x}(x+\sqrt{x^{2}-1})$
\end{itemize}
\end{homeworkProblem}

\begin{homeworkProblem}
Assume that the approximation is $\phi^*$,hence $\phi^*-\phi=\delta$.

If $\cos\phi=0$, then $\phi=\frac{\pi}{2}$or$\frac{3\pi}{2}$, now $\frac{\cos(\phi^*)-\cos(\phi)}{\cos(\phi)}$ approaches to be infinite.

If $\cos\phi\neq0$, then $\frac{\cos(\phi^*)-\cos(\phi)}{\cos(\phi)}=\frac{\cos(\phi+\delta)-\cos(\phi)}{\cos(\phi)}=\frac{\cos\delta\cos\phi-\sin\delta\sin\phi}{\cos\phi}-1=\cos\delta-1-\sin\delta\tan\phi$

Another way, relative error is $\frac{d(\cos\phi)}{\cos\phi}=\frac{(\cos\phi)^{'}}{\cos\phi}d\phi=-\delta\tan\phi$

\end{homeworkProblem}

\begin{homeworkProblem}
The result is 6.127$\times${$10^{-13}$}
\begin{lstlisting}
#include <stdio.h>
#include <math.h>

int main(){
  double a,b;	
  double result;
  scanf("%lf %lf",&a,&b);
  result=b*b/(-a+sqrt(a*a+b*b));
  printf("%.3e",result);
  return 0;
} 
\end{lstlisting}	
\end{homeworkProblem}

\begin{homeworkProblem}
The result is $x_{1}=-2.824\times10^{11},x_{2}=1.062\times10^{-11}$
\begin{lstlisting}
#include <stdio.h>
#include <math.h>

int main(){
  double a,b,c;
  double x1,x2;
  scanf("%lf %lf %lf",&a,&b,&c);
  if(b>0){
    x1=(-b-sqrt(b*b-4*a*c))/(2*a);
    x2=c/(a*x1);
  }
  else{
    x1=(-b+sqrt(b*b-4*a*c))/(2*a);
    x2=c/(a*x1);
  }//here we ignore the condition that b==0
  printf("%.3e %.3e",x1,x2);
  return 0;
} 
\end{lstlisting}	
\end{homeworkProblem}

\begin{homeworkProblem}
The result is -0.00050024507964763210.

The equivalent expression is $P(x)=\frac{1-x^{100}}{1+x}$.

According to the expression, the result is -0.00050024507964746079.

The error is $-1.713039\times10^{-16}$.
\begin{lstlisting}       
#include <stdio.h>
double horner(int coef[],int n, double x);

int main(){
  double result;
  int i=0;
  int a[100]={0};
  for(i=0;i<100;i++){
  	if(i%2==0){
  	  a[i]=1;
  	}
  	else{
      a[i]=-1;
  	}
  }
  result=horner(a,100,1.00001);
  printf("%.20lf",result); 
  return 0;
}

double horner(int coef[],int n,double x){
  int i;
  double result;
  result=coef[n-1];
  for(i=1;i<=n-1;i++){
	result=result*x+coef[n-1-i];
  }
  return result;
}
\end{lstlisting}

\end{homeworkProblem}

\end{CJK*}
\end{document}
