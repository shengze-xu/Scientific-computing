\documentclass{article}

\usepackage{fancyhdr}
\usepackage[toc,page]{appendix}
\usepackage{extramarks}
\usepackage{amsmath}
\usepackage{amsthm}
\usepackage{amsfonts}
\usepackage{tikz}
\usepackage[plain]{algorithm}
\usepackage{algpseudocode}
\usepackage{amssymb}
\usepackage{bbm}
\usepackage{extarrows}
\usepackage{mathrsfs}
\usepackage{CJK}
\usepackage{dsfont}
\usepackage[hidelinks]{hyperref}
\usepackage{apacite}
\usepackage{multirow, booktabs}  
\usepackage{threeparttable}
\usepackage{dcolumn}
\usepackage{longtable}
\usepackage{threeparttablex}
\usepackage{tabu}
\usepackage{pdfpages}
\usepackage{float}
\usepackage{changepage}
\usepackage{mathtools}
\usepackage{listings}

\usetikzlibrary{automata,positioning}

\topmargin=-0.45in
\evensidemargin=0in
\oddsidemargin=0in
\textwidth=6.5in
\textheight=9.0in
\headsep=0.25in
\setlength{\parindent}{0em}
\linespread{1.1}

\pagestyle{myheadings}
\markboth{HW Solution}{Shengze Xu}
\chead{\hmwkClass\ : \hmwkTitle}
\rhead{\firstxmark}
\lfoot{\lastxmark}
\cfoot{\thepage}

\renewcommand\headrulewidth{0.4pt}
\renewcommand\footrulewidth{0.4pt}

\newcommand{\enterProblemHeader}[1]{
    \nobreak\extramarks{}{Problem \arabic{#1} continued on next page\ldots}\nobreak{}
    \nobreak\extramarks{Problem \arabic{#1} (continued)}{Problem \arabic{#1} continued on next page\ldots}\nobreak{}
}

\newcommand{\exitProblemHeader}[1]{
    \nobreak\extramarks{Problem \arabic{#1} (continued)}{Problem \arabic{#1} continued on next page\ldots}\nobreak{}
    \stepcounter{#1}
    \nobreak\extramarks{Problem \arabic{#1}}{}\nobreak{}
}

\setcounter{secnumdepth}{0}
\newcounter{partCounter}
\newcounter{homeworkProblemCounter}
\setcounter{homeworkProblemCounter}{1}
\nobreak\extramarks{Problem \arabic{homeworkProblemCounter}}{}\nobreak{}

\newenvironment{homeworkProblem}[1][-1]{
    \ifnum#1>0
        \setcounter{homeworkProblemCounter}{#1}
    \fi
    \section{Problem \arabic{homeworkProblemCounter}}
    \setcounter{partCounter}{1}
    \enterProblemHeader{homeworkProblemCounter}
}{
    \exitProblemHeader{homeworkProblemCounter}
}

\newcommand{\hmwkTitle}{HW10 Solution}
\newcommand{\hmwkDueDate}{\today}
\newcommand{\hmwkClass}{Scientific Computing}
\newcommand{\hmwkClassInstructor}{Professor Lai}
\newcommand{\hmwkAuthorName}{Xu Shengze 3190102721}

\title{
    \vspace{2in}
    \textmd{\textbf{\hmwkClass:\ \hmwkTitle}}\\
    \normalsize\vspace{0.1in}\small{\hmwkDueDate }\\
    \vspace{0.1in}\large{\textit{\hmwkClassInstructor\ }}
    \vspace{3in}
}

\author{\textbf{\hmwkAuthorName}}
\date{}

\renewcommand{\part}[1]{\textbf{\large Part \Alph{partCounter}}\stepcounter{partCounter}\\}

% Various Helper Commands
% Useful for algorithms
\newcommand{\alg}[1]{\textsc{\bfseries \footnotesize #1}}
% For derivatives
\newcommand{\deriv}[1]{\frac{\mathrm{d}}{\mathrm{d}x} (#1)}
% For partial derivatives
\newcommand{\pderiv}[2]{\frac{\partial}{\partial #1} (#2)}
% Integral dx
\newcommand{\dx}{\mathrm{d}x}
% Alias for the Solution section header
\newcommand{\solution}{\textbf{\large Solution}}
% Probability commands: Expectation, Variance, Covariance, Bias
\newcommand{\E}{\mathrm{E}}
\newcommand{\Var}{\mathrm{Var}}
\newcommand{\Cov}{\mathrm{Cov}}
\newcommand{\Bias}{\mathrm{Bias}}

\lstset{
	basicstyle=\tt,%行号
	numbers=left,
	rulesepcolor=\color{red!20!green!20!blue!20},
	escapeinside=``,
	xleftmargin=2em,xrightmargin=2em, aboveskip=1em,%背景框
	framexleftmargin=1.5mm,
	frame=shadowbox,%背景色
	backgroundcolor=\color[RGB]{245,245,244},%样式
	keywordstyle=\color{blue}\bfseries,
	identifierstyle=\bf,
	numberstyle=\color[RGB]{0,192,192},
	commentstyle=\it\color[RGB]{96,96,96},
	stringstyle=\rmfamily\slshape\color[RGB]{128,0,0},%显示空格
	showstringspaces=false
}

\begin{document}
\lstset{language=MATLAB}
\maketitle

\pagebreak

\begin{homeworkProblem}
(1)Strictly diagonal dominance:

Assume that $A$ is irreversible, then $det(A)=0$, so $AX=0$ has a non-zero solution, suppose that $X=(x_1,x_2,\cdots,x_n)^T$, $|x_k|=max{|x_i|}$.

According to the condition, we have $\sum_{j=1}^{n}a_{kj}x_j=0$, so $|a_{kk}||x_k|=|\sum_{j\neq k}a_{kj}x_j|$.

Meanwhile, $A$ is Strict Diagonal Dominance Matrix, so $|a_{kk}||x_k|\geq|x_k|\sum_{j\neq k}|a_{kj}|>\sum_{j\neq k}|a_{kj}||x_j|\geq|\sum_{j\neq k}a_{kj}x_j|$. It contradicts, so $A$ is reversible.

(2)Irreducible diagonal advantage:

Assume that $A$ is irreversible, then $det(A)=0$, so $AX=0$ has a non-zero solution, suppose that $X=(x_1,x_2,\cdots,x_n)^T$, $K=\{k|\forall i, s.t. |x_k|\geq|x_i|, \exists j, s.t. |x_k|>|x_j|\}$.

If $K=\emptyset$, so $\forall i$, $|x_k|\geq |x_i|$ and $|x_k|\leq |x_i|$, then $|x_k|=|x_i|$, Therefore, we have $|x_1|=|x_2|=\cdots=|x_n|$.

For $\sum_{j=1}^{n}a_{kj}x_j=0$, so $|a_{kk}x_k|=|\sum_{j\neq k}a_{kj}x_j|$, then $|a_{kk}||x_k|\leq\sum_{j\neq k}|a_{kj}x_j|$, but it contradicts with the fact that $A$ is diagonally dominant. Therefore, $K\neq\emptyset$.

For $|a_{kk}||x_k|\leq\sum_{j\neq k}|a_{kj}x_j|$, we have $|a_{kk}|\leq \sum_{j\neq k}|a_{kj}|\frac{x_j}{x_k}$. Meanwhile we have $|a_{kk}|\geq \sum_{j\neq k}|a_{kj}|$, so $\sum_{j\neq k}|a_{kj}|\leq \sum_{j\neq k}|a_{kj}|\frac{x_j}{x_k}$. If $|x_j|<|x_k|$, we have $a_{kj}=0$, but according to this condition, we can easily know that $A$ is reducible. This contradicts the conditions of the title. So the assumption is wrong, $A$ is reversible.
\end{homeworkProblem}

\begin{homeworkProblem}
According to $A=D-L-L^T$, we have $B_{G-S}=(D-L)^{-1}L^T$.

Suppose $\lambda$ is one of the eigenvalues of $B_{G-S}$, $x$ is the corresponding feature vector, then we have $(D-L)^{-1}L^Tx=\lambda x$, so $L^Tx=\lambda (D-L)x$, so $x^TL^Tx=\lambda x^T(D-L)x$.

$A$ is positive definite matrix, so $p=x^TDx>0$, let $x^TL^Tx=a$, then $x^TAx=x^T(D-L-L^T)x=p-a-a=p-2a>0$.

Then we have $\lambda=\frac{x^TL^Tx}{x^T(D-L)x}=\frac{a}{p-a}$, so $\lambda^2=\frac{a^2}{(p-a)^2}=\frac{a^2}{p(p-2a)+a^2}<1$.

Therefore, spectral radius $\rho(B_{G-S})<1$, we know that Gauss–Seidel method must converge.
\end{homeworkProblem}

\begin{homeworkProblem}
Write the code of Jacobi iteration method.
\begin{lstlisting}
function [X,Result]=Jacobi(A,b,X0,Norm,epsilon,Max)
	a=[];x=[];[N N]=size(A);X=X0;
	[L,D,U]=LUD(A);
	B=eye(N)-inv(D)*A;
	d=inv(D)*b;
	X1=A\b;
	Result=Ifconverge(B);
	for i=1:Max
		X=B*X+d;
		err=norm(X-X1,Norm);
		a(i)=err;
		x=i;
		if err<epsilon
			return
		end
	end
end
\end{lstlisting}
Write the code of Gauss-Seidel iterative method.
\begin{lstlisting}
function [X,Result]=Gauss_Seidel(A,b,X0,Norm,epsilon,Max)
	a=[];x=[];[N N]=size(A);X=X0;
	[L,D,U]=LUD(A);
	B=-inv(D+L)*U;
	d=inv(D+L)*b;
	X1=A\b;
	Result=Ifconverge(B);
	for i=1:Max
		X=B*X+d;
		err=norm(X-X1,Norm);
		a(i)=err;
		x=i;
		if err<epsilon
			return
		end
	end
end
\end{lstlisting}
Write the function of matrix factorization.
\begin{lstlisting}
function [L U D]=LUD(A)
	[n m]=size(A);
	L=zeros(size(A));
	U=zeros(size(A));
	D=zeros(size(A));
	for i=1:n-1
		L(i+1:end,i)=A(i+1:end,i);
		U(i,i)=A(i,i);
		D(i,i+1:end)=A(i,i+1:end);
	end
	U(n,n)=A(n,n);
end
\end{lstlisting}
Write a function to determine whether to converge.
\begin{lstlisting}
function Result=Ifconverge(B)
	syms k;
	l=length(B);
	L=zeros(size(B));
	for i=1:l
		L(i)=limit(B(i)^k,k,inf);
	end
	if L==0
		Result=1;
	else
		Result=0;
	end
end
\end{lstlisting}
Substituting the question data, we get the corresponding results and list them in the following table.
\begin{table}[!h]
	\begin{tabular}{|c|c|c|c|c|}
		\hline
		& \multicolumn{2}{c|}{Jacobi}                                                                                                      & \multicolumn{2}{c|}{Gauss-Seidel}                                                                                        \\ \hline
		Equations 1 & \begin{tabular}[c]{@{}c@{}}{[}-0.9999999997087035,\\ -3.999999999594850,\\ -2.999999999534157{]}\end{tabular} & Converge         & \begin{tabular}[c]{@{}c@{}}{[}-0.9999999998404607,\\ -3.999999999875405,\\ -2.999999999928966{]}\end{tabular} & Converge \\ \hline
		Equations 2 & \begin{tabular}[c]{@{}c@{}}{[}-0.666666666666666,\\ -0.6666666666666666,\\ 1.333333333333333{]}\end{tabular}  & Doesn't converge & \begin{tabular}[c]{@{}c@{}}{[}0.7499999999808467,\\ 0.7500000005506776,\\ 1.250000000265762{]}\end{tabular}   & Converge \\ \hline
	\end{tabular}
\end{table}
\end{homeworkProblem}

\begin{homeworkProblem}
First, write the corresponding function program according to the power method.
\begin{lstlisting}
function [c,y]=	Power(A,x0,eps,N) 
	k=1;                       
	z=0;                
	y=x0./max(abs(x0)); 
	x=A*y;            
	xmax=max(x);         
	if abs(z-xmax)<eps      
		c=max(x);
		return;
	end
	while abs(z-xmax)>eps && k<N
		k=k+1;
		z=xmax;
		y=x./max(abs(x));   
		x=A*y;
		xmax=max(x);     
	end
	[m,i]=max(abs(x));   
	c=x(i);
end
\end{lstlisting}
Next, use the function to find the unknown quantity required by the problem.
\begin{lstlisting}
A=[1 -1 0
   -2 4 -2
   0 -1 1];
B=[2 -1 0
   -1 0 2
   1 1 3];
x0=[1;0;0];
eps=1e-9;
N=10000;
\end{lstlisting}
By running the program, we get the evaluated value as follow.

(1) $\lambda=5$, $v=[0.25,-1,0.25]^T$.

(2) $\lambda=3.000299940011996$, $v=[-0.999400119976008,0.9997000599880039,1]^T$.

\end{homeworkProblem}

\begin{homeworkProblem}
First, write the corresponding function program according to the inverse power method.
\begin{lstlisting}
function [c,y]=Inversepower(A,x0,eps,N) 
	k=1; r=0;  
	y=x0./max(abs(x0)); 
	[L,U]=lu(A);
	z=L\y;
	x=U\z;
	xmax=max(x);
	c=1/xmax;         
	if abs(xmax-r)<eps  
		return
	end
	while abs(xmax-r)>eps && k<N
		k=k+1;
		r=xmax;
		y=x./max(abs(x));
		z=L\y;
		x=U\z;
		xmax=max(x);   
	end
	[m,i]=max(abs(x)); 
	c=1/x(i); 
end
\end{lstlisting}
Next, use the function to find the unknown quantity required by the problem.
\begin{lstlisting}
format long e
A=[-4 14 0
   -5 13 0
   -1 0 2];
x0=[1;1;1];
eps=1e-9;
N=10000;
[eigenvalue,eigenvector]=Inversepower(A,x0,eps,N);
disp(eigenvalue);
disp(eigenvector);
\end{lstlisting}
By running the program, we get the evaluated value as follow.

$\lambda=2.000000005576279$, $v=[-8.364418839284320\times10^{-9},-4.182209419642158\times10^{-9},1]^T$.
\end{homeworkProblem}
\end{document}
