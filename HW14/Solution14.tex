\documentclass{article}

\usepackage{fancyhdr}
\usepackage[toc,page]{appendix}
\usepackage{extramarks}
\usepackage{amsmath}
\usepackage{amsthm}
\usepackage{amsfonts}
\usepackage{tikz}
\usepackage[plain]{algorithm}
\usepackage{algpseudocode}
\usepackage{amssymb}
\usepackage{bbm}
\usepackage{extarrows}
\usepackage{mathrsfs}
\usepackage{CJK}
\usepackage{dsfont}
\usepackage[hidelinks]{hyperref}
\usepackage{apacite}
\usepackage{multirow, booktabs}  
\usepackage{threeparttable}
\usepackage{dcolumn}
\usepackage{longtable}
\usepackage{threeparttablex}
\usepackage{tabu}
\usepackage{pdfpages}
\usepackage{float}
\usepackage{changepage}
\usepackage{mathtools}
\usepackage{listings}

\usetikzlibrary{automata,positioning}

\topmargin=-0.45in
\evensidemargin=0in
\oddsidemargin=0in
\textwidth=6.5in
\textheight=9.0in
\headsep=0.25in
\setlength{\parindent}{0em}
\linespread{1.1}

\pagestyle{myheadings}
\markboth{HW Solution}{Shengze Xu}
\chead{\hmwkClass\ : \hmwkTitle}
\rhead{\firstxmark}
\lfoot{\lastxmark}
\cfoot{\thepage}

\renewcommand\headrulewidth{0.4pt}
\renewcommand\footrulewidth{0.4pt}

\newcommand{\enterProblemHeader}[1]{
    \nobreak\extramarks{}{Problem \arabic{#1} continued on next page\ldots}\nobreak{}
    \nobreak\extramarks{Problem \arabic{#1} (continued)}{Problem \arabic{#1} continued on next page\ldots}\nobreak{}
}

\newcommand{\exitProblemHeader}[1]{
    \nobreak\extramarks{Problem \arabic{#1} (continued)}{Problem \arabic{#1} continued on next page\ldots}\nobreak{}
    \stepcounter{#1}
    \nobreak\extramarks{Problem \arabic{#1}}{}\nobreak{}
}

\setcounter{secnumdepth}{0}
\newcounter{partCounter}
\newcounter{homeworkProblemCounter}
\setcounter{homeworkProblemCounter}{1}
\nobreak\extramarks{Problem \arabic{homeworkProblemCounter}}{}\nobreak{}

\newenvironment{homeworkProblem}[1][-1]{
    \ifnum#1>0
        \setcounter{homeworkProblemCounter}{#1}
    \fi
    \section{Problem \arabic{homeworkProblemCounter}}
    \setcounter{partCounter}{1}
    \enterProblemHeader{homeworkProblemCounter}
}{
    \exitProblemHeader{homeworkProblemCounter}
}

\newcommand{\hmwkTitle}{HW14 Solution}
\newcommand{\hmwkDueDate}{\today}
\newcommand{\hmwkClass}{Scientific Computing}
\newcommand{\hmwkClassInstructor}{Professor Lai}
\newcommand{\hmwkAuthorName}{Xu Shengze 3190102721}

\title{
    \vspace{2in}
    \textmd{\textbf{\hmwkClass:\ \hmwkTitle}}\\
    \normalsize\vspace{0.1in}\small{\hmwkDueDate }\\
    \vspace{0.1in}\large{\textit{\hmwkClassInstructor\ }}
    \vspace{3in}
}

\author{\textbf{\hmwkAuthorName}}
\date{}

\renewcommand{\part}[1]{\textbf{\large Part \Alph{partCounter}}\stepcounter{partCounter}\\}

% Various Helper Commands
% Useful for algorithms
\newcommand{\alg}[1]{\textsc{\bfseries \footnotesize #1}}
% For derivatives
\newcommand{\deriv}[1]{\frac{\mathrm{d}}{\mathrm{d}x} (#1)}
% For partial derivatives
\newcommand{\pderiv}[2]{\frac{\partial}{\partial #1} (#2)}
% Integral dx
\newcommand{\dx}{\mathrm{d}x}
% Alias for the Solution section header
\newcommand{\solution}{\textbf{\large Solution}}
% Probability commands: Expectation, Variance, Covariance, Bias
\newcommand{\E}{\mathrm{E}}
\newcommand{\Var}{\mathrm{Var}}
\newcommand{\Cov}{\mathrm{Cov}}
\newcommand{\Bias}{\mathrm{Bias}}

\lstset{
	basicstyle=\tt,%行号
	numbers=left,
	rulesepcolor=\color{red!20!green!20!blue!20},
	escapeinside=``,
	xleftmargin=2em,xrightmargin=2em, aboveskip=1em,%背景框
	framexleftmargin=1.5mm,
	frame=shadowbox,%背景色
	backgroundcolor=\color[RGB]{245,245,244},%样式
	keywordstyle=\color{blue}\bfseries,
	identifierstyle=\bf,
	numberstyle=\color[RGB]{0,192,192},
	commentstyle=\it\color[RGB]{96,96,96},
	stringstyle=\rmfamily\slshape\color[RGB]{128,0,0},%显示空格
	showstringspaces=false
}

\begin{document}
\lstset{language=MATLAB}
\maketitle
	
\pagebreak
	
\begin{homeworkProblem}
Trapezoidal formula is $y_{n+1}=y_n+\frac{h}{2}(f(x_n,y_n)+f(x_{n+1},y_{n+1}))$.

We know that $T_{n+1}=y(x_{n+1})-[y(x_n)+\frac{h}{2}(f(x_n,y(x_n))+f(x_{n+1},y(x_{n+1})))]$.

According to Taylor expansion, we get the following relationship,
\begin{equation}
\left\{
\begin{aligned}
y(x_{n+1})&=y(x_n)+hy'(x_n)+\frac{h^2}{2}y''(x_n)+\frac{h^3}{6}y'''(\xi_{n_1})\\
y(x_n)&=y(x_{n+1})-hy'(x_{n+1})+\frac{h^2}{2}y''(x_{n+1})-\frac{h^3}{6}y'''(\xi_{n_2})
\end{aligned}
\right.
\end{equation}

We have $T_{n+1}=y(x_{n+1})-y(x_n)+\frac{h}{2}(y'(x_n)+y'(x_{n+1}))$, substitute the relationship of the above equations and we know that $T_{n+1}=\frac{h^3}{12}[y'''(\xi_{n_1})+y'''(\xi_{n_2})]+\frac{h^2}{4}[y''(x_{n})-y''(x_{n+1})]$.

Meanwhile, we have $y''(x_{n})-y''(x_{n+1})=-hy'''(\xi_{n_3})$, so we have $T_{n+1}=\frac{h^3}{12}[y'''(\xi_{n_1})+y'''(\xi_{n_2})-3y'''(\xi_{n_3})]$, and there must be $\xi_n$ satisfying that $y'''(\xi_n)=-(y'''(\xi_{n_1})+y'''(\xi_{n_2})-3y'''(\xi_{n_3}))$.

Finally, we have $T_{n+1}=-\frac{h^3}{12}y'''(\xi_n)$, where $\xi_n\in[x_n,x_{n+1}]$.
\end{homeworkProblem}
	
\begin{homeworkProblem}
\begin{itemize}
\item[(a)]According to the question, we have $\frac{dy}{3-2y}=dx$, integrate both sides and we have $-\frac{1}{2}\ln|3-2y|=x+c_1$. Meanwhile, we have $y(1)=2$, so $c_1=-1$, we simplify to get that  $y=\frac{3}{2}+\frac{1}{2}e^{-2(x-1)}$.

\item[(b)]We have $y_{n+1}=y_n+\frac{h}{2}(f(x_n,y_n)+f(x_{n+1},y_{n+1}))$ and $f(x_n,y_n)=3-2y_n$, so $y_{n+1}=y_n+\frac{h}{2}(3-2y_n+3-2y_{n+1})$, then $(1+h)y_{n+1}=(1-h)y_n+3h$, we simplify to get that $y_{n+1}=\frac{1-h}{1+h}y_n+\frac{3h}{1+h}$.

\item[(c)]We have $f(x,y)=3-2y$. The truncation error of step $n+1$ is $|y(x_{n+1})-y_{n+1}|\leq|y(x_n)-y_n|+\frac{h}{2}|f(x_n,y(x_n))-f(x_n,y_n)|+\frac{h}{2}|f(x_{n+1},y(x_{n+1}))-f(x_{n+1},y_{n+1})|+|T_{n+1}|\leq|y(x_n)-y_n|+\frac{h}{2}\cdot2|y(x_n)-y_n|+\frac{h}{2}\cdot2|y(x_{n+1})-y_{n+1}|+|T_{n+1}|$.

Assume that $|y(x_n)-y_n|=e_n$ and we have $y_{n+1}\leq\frac{1+h}{1-h}y_n+|T_{n+1}|$, so we have the following formula, $e_{n+1}\leq\frac{1+h}{1-h}e_n+\frac{|T_{n+1}|}{1-h}\leq\frac{1+h}{1-h}e_n+\frac{1}{1-h}\frac{h^3}{12}M$, where $M$ is the upper bound of $y'''(\xi_n)$.

Therefore, we know that  $e_n\leq\frac{1+h}{1-h}e_{n-1}+\frac{1}{1-h}\frac{h^3}{12}M\leq(\frac{1+h}{1-h})^ne_0+\frac{1}{1-h}\frac{h^3}{12}M(1+(\frac{1+h}{1-h})^1+(\frac{1+h}{1-h})^2+\cdots+(\frac{1+h}{1-h})^{n-1})$. We can easily know that $e_0=0$, so simplify the above formula and we know that $e_n\leq\frac{h^2}{24}M(e^{\frac{2h}{1-h}(n-1)}-1)\to\frac{h^2}{24} M(e^{\frac{2}{1-h}}-1)$. 

When $h\to0$, we have $e_n\to0$, further, $y_n\to y(x)$.

\item[(d)]According to the principle, write the following program.
\begin{lstlisting}
format long e;
for m=1:5 
	h=10^(-m); 
	y=2; 
	for i=1:10^m 
		y=(1-h)*y/(1+h)+3*h/(1+h);
	end
	real=3/2+1/2*exp(-2);
	disp(m);
	disp(y); 
	disp(y-real); 
end
\end{lstlisting}
Run the program, we get the following results.

$m:1$, calculated value:$1.567215316374655$, error:$-4.523252436512415\times10^{-4}$.

$m:2$, calculated value:$1.567663130321896$, error:$-4.511296410658616\times10^{-6}$.

$m:3$, calculated value:$1.567667596506614$, error:$-4.511169260368320\times10^{-8}$.

$m:4$, calculated value:$1.567667641167568$, error:$-4.507381134999378\times10^{-10}$.

$m:5$, calculated value:$1.567667641614879$, error:$-3.427702566227708\times10^{-12}$.
\end{itemize}
\end{homeworkProblem}
	

\end{document}
