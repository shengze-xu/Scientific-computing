\documentclass{article}

\usepackage{fancyhdr}
\usepackage[toc,page]{appendix}
\usepackage{extramarks}
\usepackage{amsmath}
\usepackage{amsthm}
\usepackage{amsfonts}
\usepackage{tikz}
\usepackage[plain]{algorithm}
\usepackage{algpseudocode}
\usepackage{amssymb}
\usepackage{bbm}
\usepackage{extarrows}
\usepackage{mathrsfs}
\usepackage{CJK}
\usepackage{dsfont}
\usepackage[hidelinks]{hyperref}
\usepackage{apacite}
\usepackage{multirow, booktabs}  
\usepackage{threeparttable}
\usepackage{dcolumn}
\usepackage{longtable}
\usepackage{threeparttablex}
\usepackage{tabu}
\usepackage{pdfpages}
\usepackage{float}
\usepackage{changepage}
\usepackage{mathtools}
\usepackage{listings}

\usetikzlibrary{automata,positioning}

\topmargin=-0.45in
\evensidemargin=0in
\oddsidemargin=0in
\textwidth=6.5in
\textheight=9.0in
\headsep=0.25in
\setlength{\parindent}{0em}
\linespread{1.1}

\pagestyle{myheadings}
\markboth{HW Solution}{Shengze Xu}
\chead{\hmwkClass\ : \hmwkTitle}
\rhead{\firstxmark}
\lfoot{\lastxmark}
\cfoot{\thepage}

\renewcommand\headrulewidth{0.4pt}
\renewcommand\footrulewidth{0.4pt}

\newcommand{\enterProblemHeader}[1]{
    \nobreak\extramarks{}{Problem \arabic{#1} continued on next page\ldots}\nobreak{}
    \nobreak\extramarks{Problem \arabic{#1} (continued)}{Problem \arabic{#1} continued on next page\ldots}\nobreak{}
}

\newcommand{\exitProblemHeader}[1]{
    \nobreak\extramarks{Problem \arabic{#1} (continued)}{Problem \arabic{#1} continued on next page\ldots}\nobreak{}
    \stepcounter{#1}
    \nobreak\extramarks{Problem \arabic{#1}}{}\nobreak{}
}

\setcounter{secnumdepth}{0}
\newcounter{partCounter}
\newcounter{homeworkProblemCounter}
\setcounter{homeworkProblemCounter}{1}
\nobreak\extramarks{Problem \arabic{homeworkProblemCounter}}{}\nobreak{}

\newenvironment{homeworkProblem}[1][-1]{
    \ifnum#1>0
        \setcounter{homeworkProblemCounter}{#1}
    \fi
    \section{Problem \arabic{homeworkProblemCounter}}
    \setcounter{partCounter}{1}
    \enterProblemHeader{homeworkProblemCounter}
}{
    \exitProblemHeader{homeworkProblemCounter}
}

\newcommand{\hmwkTitle}{HW5 Solution}
\newcommand{\hmwkDueDate}{\today}
\newcommand{\hmwkClass}{Scientific Computing}
\newcommand{\hmwkClassInstructor}{Professor Lai}
\newcommand{\hmwkAuthorName}{Xu Shengze 3190102721}

\title{
    \vspace{2in}
    \textmd{\textbf{\hmwkClass:\ \hmwkTitle}}\\
    \normalsize\vspace{0.1in}\small{\hmwkDueDate }\\
    \vspace{0.1in}\large{\textit{\hmwkClassInstructor\ }}
    \vspace{3in}
}

\author{\textbf{\hmwkAuthorName}}
\date{}

\renewcommand{\part}[1]{\textbf{\large Part \Alph{partCounter}}\stepcounter{partCounter}\\}

% Various Helper Commands
% Useful for algorithms
\newcommand{\alg}[1]{\textsc{\bfseries \footnotesize #1}}
% For derivatives
\newcommand{\deriv}[1]{\frac{\mathrm{d}}{\mathrm{d}x} (#1)}
% For partial derivatives
\newcommand{\pderiv}[2]{\frac{\partial}{\partial #1} (#2)}
% Integral dx
\newcommand{\dx}{\mathrm{d}x}
% Alias for the Solution section header
\newcommand{\solution}{\textbf{\large Solution}}
% Probability commands: Expectation, Variance, Covariance, Bias
\newcommand{\E}{\mathrm{E}}
\newcommand{\Var}{\mathrm{Var}}
\newcommand{\Cov}{\mathrm{Cov}}
\newcommand{\Bias}{\mathrm{Bias}}

\lstset{
	basicstyle=\tt,%行号
	numbers=left,
	rulesepcolor=\color{red!20!green!20!blue!20},
	escapeinside=``,
	xleftmargin=2em,xrightmargin=2em, aboveskip=1em,%背景框
	framexleftmargin=1.5mm,
	frame=shadowbox,%背景色
	backgroundcolor=\color[RGB]{245,245,244},%样式
	keywordstyle=\color{blue}\bfseries,
	identifierstyle=\bf,
	numberstyle=\color[RGB]{0,192,192},
	commentstyle=\it\color[RGB]{96,96,96},
	stringstyle=\rmfamily\slshape\color[RGB]{128,0,0},%显示空格
	showstringspaces=false
}

\begin{document}
\begin{CJK*}{GBK}{song}
\lstset{language=MATLAB}
\maketitle

\pagebreak

\begin{homeworkProblem}
According to the coefficient formula, we know that $C_{n-i}^{(n)}=\frac{(-1)^{i}}{n\cdot{i!}\cdot(n-i)!}\int_{0}^{n}\frac{t(t-1)\cdot\cdot\cdot(t-n)}{t-(n-i)}dt$.

Let $t=n-x$, then $C_{n-i}^{(n)}=\frac{(-1)^{i}}{n\cdot{i!}\cdot(n-i)!}\int_{n}^{0}-\frac{(n-x)(n-x-1)\cdot\cdot\cdot(-x)}{(n-x)-(n-i)}dx=\frac{(-1)^{i}}{n\cdot{i!}\cdot(n-i)!}\cdot(-1)^{n}\int_{0}^{n}\frac{x(x-1)\cdot\cdot\cdot(x-n)}{x-i}dx$.

So $C_{n-i}^{(n)}=\frac{(-1)^{n+i}}{n\cdot{i!}\cdot(n-i)!}\int_{0}^{n}\frac{t(t-1)\cdot\cdot\cdot(t-n)}{t-i}dt$, and we know that $C_{i}^{(n)}=\frac{(-1)^{n-i}}{n\cdot{i!}\cdot(n-i)!}\int_{0}^{n}\frac{t(t-1)\cdot\cdot\cdot(t-n)}{t-i}dt$.

It's obvious that $(-1)^{n+i}=(-1)^{n-i}$, therefore $C_{i}^{(n)}=C_{n-i}^{(n)}$, $i=0,1,2,\cdot\cdot\cdot,n$.
\end{homeworkProblem}

\begin{homeworkProblem}
According to Taylor's theorem, we have $f(x)=f(\frac{a+b}{2})+(x-\frac{a+b}{2})f'(\frac{a+b}{2})+\frac{(x-\frac{a+b}{2})^2}{2}f''(\xi)$, where $\xi\in(a,b)$.

So $\int_{a}^{b}f(x)dx=\int_{a}^{b}f(\frac{a+b}{2})dx+\int_{a}^{b}(x-\frac{a+b}{2})f'(\frac{a+b}{2})dx+\int_{a}^{b}\frac{(x-\frac{a+b}{2})^2}{2}f''(\xi)dx$, where $\xi\in(a,b)$.

And we could easily konw that,  $\int_{a}^{b}(x-\frac{a+b}{2})f'(\frac{a+b}{2})dx=f'(\frac{a+b}{2})\int_{a}^{b}(x-\frac{a+b}{2})dx=0$.

Therefore, $\int_{a}^{b}f(x)dx=f(\frac{a+b}{2})+\int_{a}^{b}\frac{(x-\frac{a+b}{2})^2}{2}f''(\xi)dx=f(\frac{a+b}{2})+\frac{(b-a)^3}{24}f''(\xi)$, where $\xi\in(a,b)$.	
\end{homeworkProblem}

\begin{homeworkProblem}
$I_2(f)=\int_{-1}^{1}f(x)dx=A_1f(-1)+A_2f(-\frac{1}{3})+A_3f(\frac{1}{3})$.

Take $f(x)$ as $1,x,x^2$, then we get the relation below,
\begin{equation}
	\left\{
	\begin{aligned}
		2&=A_1+A_2+A_3\\
		0&=-A_1-\frac{1}{3}A_2+\frac{1}{3}A_3\\
		\frac{2}{3}&=A_1+\frac{1}{9}A_2+\frac{1}{9}A_3
	\end{aligned}
	\right.
\end{equation}
From the equation set, we can get that
\begin{equation}
	\left\{
	\begin{aligned}
		A_1&=\frac{1}{2}\\
		A_2&=0\\
		A_3&=\frac{3}{2}
	\end{aligned}
	\right.
\end{equation}
\end{homeworkProblem}

\begin{homeworkProblem}
$I_2(f)=\frac{1}{3}(f(-1)+2f(x_1)+3f(x_2))$.

Take $f(x)$ as $1,x,x^2$, then we get the relation below,
\begin{equation}
	\left\{
	\begin{aligned}
		2&=\frac{1}{3}\times6\\
		0&=\frac{1}{3}(-1+2x_1+3x_2)\\
		\frac{2}{3}&=\frac{1}{3}(1+2x_1^2+3x_2^2)
	\end{aligned}
	\right.
\end{equation}
From the equation set, we can get that
\begin{equation}
	\left\{
	\begin{aligned}
		x_1&=\frac{1\pm\sqrt{6}}{5}\\
		x_2&=\frac{3\mp2\sqrt{6}}{15}
	\end{aligned}
	\right.
\end{equation}
\end{homeworkProblem}

\begin{homeworkProblem}
Firstly, we define two functions.
\begin{lstlisting}
function ans= trap(f,a,b)
	ans=1/2*(b-a)*(f(a)+f(b));
end
\end{lstlisting}
\begin{lstlisting}
function ans= simpson(f,a,b)
	ans=1/6*(b-a)*(f(a)+4*f((a+b)/2)+f(b));
end
\end{lstlisting}
\begin{itemize}
\item[(a)]
\begin{lstlisting}
format long e
syms x1;
f1=exp(x1);
f=inline('exp(x)','x');
ans1=trap(f,1.1,1.8);
ans2=simpson(f,1.1,1.8);
true=int(f1,x1,1.1,1.8);
disp(ans1);
disp(ans2);
disp(vpa(true));
disp(vpa(true-ans1));
disp(vpa(true-ans2));
\end{lstlisting}
Trapezoid:$3.168834720925783$

Simpson:$3.045731680720709$

Eaxct value:$3.045481440466512$
\item[(b)]
\begin{lstlisting}
syms x2;
f2=(sin(x2))^2;
f=inline('(sin(x))^2','x');
ans1=trap(f,0,pi/2);
ans2=simpson(f,0,pi/2);
true=int(f2,x2,0,pi/2);
disp(ans1);
disp(ans2);
disp(vpa(true));
disp(vpa(true-ans1));
disp(vpa(true-ans2));
\end{lstlisting}
Trapezoid:$7.853981633974483\times10^{-1}$

Simpson:$7.853981633974481\times10^{-1}$

Eaxct value:$7.85398163397448309\times10^{-1}$
\end{itemize}
\begin{itemize}
\item[(a)]
\begin{lstlisting}
format long e
syms x3;
f3=exp(-(x3)^2);
f=inline('exp(-x^2)','x');
ans1=trap(f,1,2);
ans2=simpson(f,1,2);
disp(ans1);
disp(ans2);
error1=1/12*max(abs(subs(diff(f3,x3,2),x3,[1:0.01:2])));
error2=1/2880*max(abs(subs(diff(f3,x3,4),x3,[1:0.01:2])));
fprintf('%e\n',error1);
fprintf('\n');
fprintf('%e\n',error2);
\end{lstlisting}
Trapezoid:$1.930975400300882\times10^{-1}$

Simpson:$1.346319963846057\times10^{-1}$

Trapezoid-error:$7.437168\times10^{-2}$

Simpson-error:$2.554718\times10^{-3}$
\item[(b)]
\begin{lstlisting}
syms x4;
f4=sin(x4)/(x4);
f=inline('sin(x)/x','x');
ans1=pi/2*(1+f(pi/2))/2
ans2=pi/12*(1+4*f(pi/4)+f(pi/2))
disp(ans1);
disp(ans2);
error1=1/12*(pi/2)^3*max(abs(subs(diff(f4,x4,2),x4,[0.01:0.01:pi/2])));
error2=1/2880*(pi/2)^5*max(abs(subs(diff(f4,x4,4),x4,[0.01:0.01:pi/2])));
fprintf('%e\n',error1);
fprintf('\n');
fprintf('%e\n',error2);
\end{lstlisting}
Trapezoid:$1.285398163397448$

Simpson:$1.371275096047879$

Trapezoid-error:$1.076575\times10^{-1}$

Simpson-error:$6.640815\times10^{-4}$
\end{itemize}
\end{homeworkProblem}

\end{CJK*}
\end{document}
